\documentclass[french]{article}
\usepackage[utf8x]{inputenc}
\usepackage[T1]{fontenc}
\usepackage{babel}
\usepackage{lmodern}
\usepackage[top=2cm,bottom=2cm,left=3cm,right=3cm]{geometry}
\usepackage{microtype}
\usepackage{mathtools, amssymb, amsthm}
\usepackage{mdframed}
%\usepackage{hyperref}
\usepackage{graphicx}
\usepackage{xcolor}
\usepackage{mathrsfs}
\usepackage{wrapfig}
\usepackage{stmaryrd}
\usepackage{dsfont}
\usepackage{framed}
\usepackage[Glenn]{fncychap}
%\usepackage{poemscol}
\usepackage{verse}

\newtheorem{prototheorem}{Théorème}[section]
\newenvironment{thm}
   {\colorlet{shadecolor}{orange!10}\begin{shaded}\begin{prototheorem}}
   {\end{prototheorem}\end{shaded}}

\newtheorem*{protocorollary}{Corollaire}
\newenvironment{corollary}
    {\colorlet{shadecolor}{violet!10}\begin{shaded}\begin{protocorollary}}
    {\end{protocorollary}\end{shaded}}

\newtheorem*{protolemma}{Lemme}
\newenvironment{lemma}
    {\colorlet{shadecolor}{pink!15}\begin{shaded}\begin{protolemma}}
    {\end{protolemma}\end{shaded}}

\theoremstyle{definition}
\newtheorem{protodefinition}{Définition}[section]
\newenvironment{definition}
    {\colorlet{shadecolor}{green!5}\begin{shaded}\begin{protodefinition}}
    {\end{protodefinition}\end{shaded}}

\newtheorem{protoproposition}{Proposition}[section]
\newenvironment{prop}
    {\colorlet{shadecolor}{blue!5}\begin{shaded}\begin{protoproposition}}
    {\end{protoproposition}\end{shaded}}

\theoremstyle{remark}
\newtheorem*{remark}{Remarque}
\newtheorem{exo}{Exercice}
\newtheorem*{protoexemple}{Exemple}
\newenvironment{exemple}
    {\colorlet{shadecolor}{gray!10}\begin{shaded}\begin{protoexemple}}
    {\end{protoexemple}\end{shaded}}


\newcommand{\lesss}{\rotatebox[origin=c]{90}{$\land$}}
\newcommand{\less}{\ \lesss\ }

\newcommand{\biggg}{\rotatebox[origin=c]{90}{$\lor$}}
\newcommand{\bg}{\ \biggg\ }

\title{\bsc{Test}}
\date{2023}

\begin{document}

\maketitle

\tableofcontents



\poemtitle{Mon pays me fait mal}

\emph{Ecrit en hommage à R. Brasillach dans un pays et un contexte différents.}

\begin{verse}

Mon pays m'a fait mal par ses multiples peines,\\
Par ses discours de miel et ses actes de fiel,\\
Par ses fils innoncents embarqués par centaines,\\
Par les villes rasées par amour fraternel.

Mon pays m'a fait mal par les fleuves de larmes,\\
Par les ``oh !'' de Moscou, les ``hourra !'' des villages,\\
Par les bonbons amers et les factices armes,\\
Par les rides creusées sur son jeune visage.

Mon pays m'a fait mal par son mythe optimiste,\\
Par ses soirs de tourments et ses matins oisifs,\\
Par son regard rieur et son sourire triste,\\
Par le grand silence, son art d'être passif.

Mon pays m'a fait mal par son deuil génétique,\\
Par les cieux et l'enfer qui sentent le pétrole,\\
Par le sang versé sur les pages des chroniques\\
Et par le réveillon qu'on passera en taule.

Mon pays m'a fait mal par les grandes parades,\\
Par les majors qui marchent sur les os des paysans,\\
Par ce gars de vingt ans au regard si maussade,\\
Par ce sexagénère au regard insouciant.

Mon pays m'a fait mal par sa fausse rancoeur,\\
Par les yeux affamés du vautour bicéphale,\\
Par les héros salauds et les bourreaux au grand coeur,\\
Par tous ces enfants nés sous la mauvaise étoile.

Mon pays m'a fait mal par le tsar qui le mine,\\
Par la neige tombant sur les corps mutilés,\\
Par son drapeau qui flotte au-dessus des ruines,\\
Par les têtes coupées pour des pensées ailées.

Mon pays m'a fait mal par ses vils patriotes,\\
Par le soleil d'hiver qui ne réchauffe rien,\\
Par ses lettres de haine et sa science idiote,\\
Par les vies échangées contre des pots-de-vin.

Mon pays m'a fait mal par sa guerre et sa paix,\\
Par son passé moqué et son futur meurtri,\\
Par les terres qu'il occupe, celles qu'il occupait.\\
Mon pays me fait mal. Quand sera-t-il guéri ?
\end{verse}

\poemtitle{La viorne obier et le faucon}

\emph{Traduit du russe par un traducteur bidon}

\begin{verse}
Une viorne obier rouge se dressait dans la cour,\\
Elle dédiait au faucon tout son amour.

Le faucon si hardi se battait dans les cieux,\\
Se posait sur ses branches juste après, silencieux.

Le faucon déploya ses ailes dans les nuages,\\
Dans ses terres natales la guerre faisait rage.

La viorne obier rouge baisse ses feuilles,\\
Et elle pleure: "Où es-tu ? Je suis tellement seule !

Je t'aurais réchauffé avec mon feuillage\\
Et sur tes ailes, avec toi, partirais dans les nuages !"

Une caille entendra ses cris étouffés\\
Et dira à la viorne d'une voix fatiguée:

"Ton faucon n'est plus là, tes plaintes sont vaines,\\
Ton ami bien-aimé est tombé en Ukraine."
\end{verse}

\end{document}
