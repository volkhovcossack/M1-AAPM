\documentclass[french]{book}
\usepackage[utf8x]{inputenc}
\usepackage[T1]{fontenc}
\usepackage{babel}
\usepackage{lmodern}
\usepackage[top=2cm,bottom=2cm,left=3cm,right=3cm]{geometry}
\usepackage{microtype}
\usepackage{mathtools, amssymb, amsthm}
\usepackage{mdframed}
\usepackage{hyperref}
\usepackage{graphicx}
\usepackage{xcolor}
\usepackage{mathrsfs}
\usepackage{wrapfig}
\usepackage{stmaryrd}

\newtheorem{prop}{Proposition}[section]
\newtheorem{theorem}{Théorème}
\newtheorem{definition}{Définition}[section]
\newtheorem*{remark}{Remarque}
\newtheorem*{lemma}{Lemme}
\newtheorem*{corollary}{Corollaire}
\newtheorem*{mth}{Méthode}
\newmdtheoremenv{thm}{Théorème}
\newtheorem{exo}{Exercice}
\newtheorem{exemple}{Exemple}


\newcommand*{\TakeFourierOrnament}[1]{{%
\fontencoding{U}\fontfamily{futs}selectfont\char#1}}
\newcommand*{\danger}{\TakeFourierOrnament{66}}

\newcommand{\lesss}{\rotatebox[origin=c]{90}{$\land$}}
\newcommand{\less}{\ \lesss\ }

\newcommand{\biggg}{\rotatebox[origin=c]{90}{$\lor$}}
\newcommand{\bg}{\ \biggg\ }

\title{\bsc{Théorie des représentations}}
\date{2023-2024}
\author{Yves \bsc{Aubry}, M-147A, yves.aubry@univ-tln.fr}

\begin{document}

\maketitle

\tableofcontents

\chapter{Généralités sur les groupes}

\section{Rappels}


Soit $G$ un groupe. Soit $H$ un sous-groupe de $G$ (i. e. $H \neq 0$ et $\forall x, y \in H, x y ^{-1} \in H$).

Considérons la relation binaire suivante sur $G$ :

Pour $x, y \in G$, $x \equiv _{d} y \mod H$ ssi $x y ^{-1} \in H$. C'est une relation d'équivalence. Elle est dite de congruence à gauche modulo $H$.

\begin{proof}
  En effet, si $x \in G$, alors $x x ^{-1} = e \in H$, donc $x \mod _{g} = x \mod H$. La relation est donc réflexive.

  De plus, si $x, y \in G$ tels que $x \equiv _{g} y \mod H$, alors $x y ^{-1} \in H$. $H$ étant un sous-groupe de $G$, il est donc stable par passage au symétrique. D'où $(x y ^{-1} ) ^{-1}  \in H$, i. e. $y x ^{-1} \in H$, c'est-à-dire $y \equiv _{g} x \mod H$.

  Enfin, si $x, y,z \in G$ tels que $x \equiv _{g} y \mod H$ et $y \equiv _{g} z \mod H$, alors $x y ^{-1} \in H$ et $yz ^{-1} \in H$. Or, $H$ étant un sous-groupe de $G$, donc $H$ est stable pour la loi de composition interne. D'où $(x y ^{-1} )(y z ^{-1} ) \in H$. Par associativité, $x (y y ^{-1} ) z ^{-1}  \in H$, ie $x z ^{-1}  \in H$.

  Donc $x \equiv _{g} z \mod H $ et la relation est transitive.
\end{proof}

\

Soit $x \in G$. La classe d'équivalence de $x$ pour cette relation d'équivalence est

\begin{gather*}
  cl _{d}(x) = \{ y \in G \mid x y ^{-1} \in H \} \\
  = \{ y \in G \mid \exists h \in H, x y ^{-1} = h \} \\
  = \{ y \in G \mid \exists h \in H, y = hx \} \\
  = \{ hx, h \in H \} =: Hx
\end{gather*}

De même, on considère, sur $G$,  la relation de congruence à gauche modulo $H$  :

\begin{equation*}
  x \equiv _{g} y \mod H \text{ ssi } x ^{-1} y \in H.
\end{equation*}

On montre de même que c'est une relation d'équivalence. Si $x \in G$, alors $cl_g(x) := xH = \{ xh, h \in H \} $.

\begin{remark}
  Si $G$ est abélien, alors les classes à gauche et à droite modulo $H$ coincident.
\end{remark}

\begin{definition}
  Un sous-groupe $H$ d'un groupe $G$ est dit distingué dans $G$ (ou normal) si :

  \begin{gather*}
    \forall  x \in G, xH = Hx, \\
    \text{ i. e. } \forall x \in G, x H x ^{-1}  \subset H \\
    \text{ i. e. } \forall x \in G, x H x ^{-1} =H.
  \end{gather*}
\end{definition}

On note alors $H \triangleleft G$.

\begin{remark}
  Tout sous-groupe d'un groupe abélien est distingué.
\end{remark}

\begin{prop}
  Soit $G$ un groupe et $H$ un sous-groupe distingué de $G$.

  On note $G/H$ l'ensemble des classes à droite ou à gauche modulo $H$.

  Si $x, y \in G$ et si l'on note $\overline{a} $ la classe de $a$ modulo $H$, on peut munir le quotient $G/H$ d'une structure de groupe en posant

  \begin{gather*}
    \overline{x} \cdot \overline{y} = \overline{xy}.
  \end{gather*}
\end{prop}

\begin{proof}
  Cette loi est bien définie, i. e. elle ne dépend pas du choix des représentants des classes d'équivalence.
\end{proof}

\begin{remark}
  Cette loi de la surjection canonique $\pi:
    \begin{array}{lll}
    G & \longrightarrow & G/H \\
    x & \longmapsto \overline{x}
    \end{array}$ un morphisme de groupes.
\end{remark}

\begin{thm}[Lagrange]
  Soit $G$ un groupe fini et $H$ un sous-groupe de $G$.

  Alors l'ordre de $H$ divise l'ordre de $G$.
\end{thm}

\begin{remark}
  L'ordre d'un groupe est simplement son cardinal.
\end{remark}

\begin{remark}
  Si $g$ est un élément de $G$, alors l'ordre de $G$ est défini comme l'ordre du sous-groupe $\langle g \rangle $ engendré par $g$. S'il est fini, alors l'ordre de $g$ est le plus petit entier $n$ tel que $g ^{n} =e$.

  D'après le théorème de Lagrange, l'ordre d'un élément divise l'ordre du groupe.
\end{remark}

\begin{remark}
  Si $G$ est un groupe fini et $H$ un sous-groupe de $G$, alors les classes (à gauche) modulo $H$ ont toutes le même cardinal, à savoir celui de $H$. En effet, l'application, pour $x \in G : f_x:
    \begin{array}{lll}
    H & \longrightarrow & xH \\
    h & \longmapsto xh
    \end{array}$ est bijective.
\end{remark}

\section{Exemples de groupes}

\subsection{$ (\mathbb{Z}, +)$ }

Groupe abélien.

$n \mathbb{Z} = \{ nk, k \in \mathbb{Z} \} $ est un sous-groupe de $\mathbb{Z}$.

\begin{remark}
  Tout sous-groupe de $\mathbb{Z}$ est de la forme $n \mathbb{Z}$ pour un certain $n \mathbb{Z}$.
\end{remark}

\subsection{$\mathbb{Z}/{ n }\mathbb{Z}$}

C'est l'ensemble des classes d'équivalence pour la relation d'équivalence suivante :

\begin{gather*}
  x, y \in \mathbb{Z}, x \equiv y \mod n \mathbb{Z} \text{ ssi } x - y \in n\mathbb{Z}.
\end{gather*}

\begin{remark}
  $\overline{x} = \overline{y}  $ ssi $x R y$.
\end{remark}

On munit l'ensemble quotient $\mathbb{Z}/{ n }\mathbb{Z}$ d'une structure de groupe (et même d'anneau) en posant, pour $x, y \in \mathbb{Z}$ : $\overline{x}+ \overline{y} = \overline{x+y}   $ (et $\overline{x} \times \overline{y} = \overline{x \times y}   $).

\begin{remark}
  $\mathbb{Z}/{ 6 }\mathbb{Z}$ anneau non intègre, car $ \overline{2} \times \overline{3} = \overline{0}   $.
\end{remark}

\begin{remark}
  $\mathbb{Z}/{ n }\mathbb{Z}$ est un corps ssi $n$ est premier.
\end{remark}

\begin{prop}
  Tous les groupes $\mathbb{Z}/{ n }\mathbb{Z}$ sont cycliques. Les générateurs sont les $\overline{a} $ tels que $a$ et $n$ sont premiers entre eux, i. e. $(a,n) = 1$. De plus, tout groupe cyclique est isomorphe à $\mathbb{Z}/{ n }\mathbb{Z}$ avec $n = \lvert G \rvert$.

  Enfin, si $G$ est cyclique d'ordre $n$ alors pour tout diviseur $d$ de $n$, $G$ admet un sous-groupe d'ordre $d$, et celui-ci est unique, et celui-ci est cyclique.
\end{prop}

\begin{remark}
  $\mathbb{Z}/{ 2 }\mathbb{Z} \times \mathbb{Z}/{ 3 }\mathbb{Z} = \{ (\overline{a}, \tilde{a} ), \overline{a} \in \mathbb{Z}/{ 2 }\mathbb{Z}, \tilde{a} \in \mathbb{Z}/{ 3 }\mathbb{Z}  \} $.
\end{remark}

\begin{thm}[Théorème des restes chinois]
  Soient $n_1, \dots, n_r$ des entiers premiers entre eux deux à deux. Alors l'application

  \[
  \begin{array}{lll}
  \mathbb{Z}/{ \prod_{i=1}^{r} n_i  }\mathbb{Z} & \longrightarrow & \prod_{i=1}^{r } \mathbb{Z}/{ n_i }\mathbb{Z}  \\
  a + (\prod_{i=1}^{r} n_i ) \mathbb{Z} & \longmapsto (a+ n_1 \mathbb{Z}, \dots, a+ n_r \mathbb{Z})
  \end{array}
  \]

  est un isomorphisme d'anneaux et la réciproque est vraie.
\end{thm}

\marginpar{19-09-2023}

\section{Groupe diédral}

Soit $n \geq 3$ un entier. Le groupe diédral de degré $n$ est le groupe des isométries du plan laissant fixe le polygone régulier à $n$ côtés. On le note $D_n$ (ou $D _{2n}$).

$D_n$ est un groupe d'ordre $2n$ constitué de $n$ rotations et de $n$ symétries.

Considérons le polygone régulier dont les sommets sont, dans le plan complexe, les $n$ racines $n$-ièmes de l'unité :

\[
e^{\frac{2ik \pi}{n}}, k = 0, 1, \dots, n-1.
\]

\begin{figure}[h!]
  \centering
  \includegraphics[scale=0.3]{figures/racines-3.jpg}
  \caption{Racines 3-ièmes de l'unité.}
  \label{}
\end{figure}

Soit $r = rot(0, \frac{2 \pi}{n})$ la rotation de centre $O$ et d'angle $\frac{2 \pi}{n}$ et soit $s$ la symétrie axiale d'axe la droite réelle $(x,x)$.

On a \[
r:\begin{array}{rcl}
\mathbb{C} & \longrightarrow & \mathbb{C} \\
z & \longmapsto e^{\frac{2 i \pi}{n}} z
\end{array}
\]

et

\[
s:
  \begin{array}{rcl}
  \mathbb{C} & \longrightarrow & \mathbb{C} \\
  z & \longmapsto \overline{z}
  \end{array}.
\]


On vérifie que l'on a $r ^{n} = 1 = id$, $s ^2 = 1 =id$ et $rs = r ^{-1} $.

\begin{proof}
  En effet, si $z \in \mathbb{C}$, alors

  \[
  r ^{-1} (z) = e^{- \frac{2 i \pi}{n}} z \text{ et } s r s(z) = s r (\overline{z} ) = s \left( e^{\frac{2 i \pi}{n}} \overline{z} \right) = e^{- \frac{2 i \pi}{n}} z = r ^{-1} (z),
  \]

  donc $s r s = r ^{-1} $.
\end{proof}

On peut donc définir le groupe diédral $D_n$ par ``générateurs et relations'' de la façon suivante :

\[
D_n = \langle r,s \rangle  \text{ avec } r ^{n} = s ^2 = 1 \text{ et } s r s = r ^{-1} .
\]

Le sous-groupe de $D_n$ engendré par $r$ est un sous-groupe d'ordre $n$ :

\[
\langle r \rangle = \{ r, r ^2, \dots, r ^{n-1}, id \} \simeq \mathbb{Z}/{ n }\mathbb{Z} .
\]

Il est d'indice 2 dans $D_n$, il est donc distingué dans $D_n$.

\subsection{Description du groupe $D_3$}

\begin{figure}[h!]
  \centering
  \includegraphics[scale=0.3]{figures/elts-d3.png}
  \caption{Description explicite des éléments de $D_3$.}
  \label{}
\end{figure}

On a donc

\[
D_3 = \{ e, r, r ^2, s, rs, r ^2 s \}
\]

\begin{remark}
  Il n'existe que deux groupes d'ordre 6 à isomorphisme près, à savoir le groupe cyclique (abélien) $\mathbb{Z}/{ 6 }\mathbb{Z}$ et le groupe symétrique (non abélien) $\mathfrak{S}_{3} $.
\end{remark}

Or $D_3$ n'est pas abélien, donc $D_3$ est isomorphe à $\mathfrak{S}_{3} $.

\begin{exo}
  Déterminer l'ordre des éléments de $D_3$ ainsi que ses sous-groupes.
\end{exo}

%\paragraph{Exemple du groupe quaternionien}

\begin{exemple}[Groupe quaternionien]
  Soit $\mathbb{H}$ le corps des quaternions d'Hamilton.

  \[
  \mathbb{H} = \{ a+ib+jc+kd \mid i ^2 = j ^2 = k ^2 = 1, ij = -ij = k, jk=-kj=i, ki=-ik=j \text{ et }  a, b, c, d \in \mathbb{R}\}.
  \]

  $\mathbb{H}$ est un corps non commutatif. On $\mathbb{R} \subset \mathbb{C} \subset \mathbb{H}$.

  Considérons le sous-ensemble suivant de $\mathbb{H}$ :

  \[
  \mathbb{H} _{8} = \{ 1, -1, i, -i, j, -j, k, -k \}.
  \]

  \begin{exo}
    Montrer que $\mathbb{H}_8$ muni de la multiplication est un groupe.
  \end{exo}

  C'est un groupe non abélien d'ordre 8.

  \begin{exo}
    Déterminer l'ordre des éléments de $\mathbb{H} _{8}$ ainsi que ses sous-groupes.
  \end{exo}
\end{exemple}



\paragraph{Rappel}


\begin{thm}[De classification des groupes abéliens finis]
  Tout groupe \textbf{abélien} fini est isomorphe à un produit de groupes cycliques de la forme

  \[
  \mathbb{Z}/{ d_1 }\mathbb{Z} \times \mathbb{Z}/{ d_2 }\mathbb{Z} \times \dots \times \mathbb{Z}/{ d_r }\mathbb{Z}, \text{ avec } d_1 \mid d_2 \mid \dots \mid d_r.
  \]

  Cette écriture est unique (à l'ordre près des facteurs).
\end{thm}

On en déduit qu'il existe trois groupes abéliens d'ordre 8 à isomorphisme près :

\[
\mathbb{Z}/{ 8 }\mathbb{Z}, \mathbb{Z}/{ 2 }\mathbb{Z} \times \mathbb{Z}/{ 4 }\mathbb{Z} \text{ et }  (\mathbb{Z}/{ 2 }\mathbb{Z}) ^3.
\]

Question : a-t-on $\mathbb{H}_8 \simeq D_4$ ?

\section{Les théorèmes de Sylow}

Si $H$ est un sous-groupe d'un groupe $G$, ses \textbf{conjugués} dans $G$ sont $g H g ^{-1} $, avec $g \in G$. En particulier, $H$ est distingué dans $G$ si et seulement si il est égal à tous ses conjugués.

\begin{definition}
  Si $G$ est un groupe fini d'ordre $p ^{\alpha} q$, avec $p$ premier, $\alpha \geq 1$ et $q$ premier avec $p$, alors tout sous-groupe de $G$ d'ordre $p3\alpha$ est appelé un $p$ sous-groupe de Sylow de $G$ (ou encore un $p$-Sylow de $G$).
\end{definition}

\begin{thm}[Premier théorème de Sylow]
  Soit $G$ un groupe d'ordre $p ^{\alpha} q$, $p$ premier, $\alpha \geq 1$, $(p, q)=1$. Pour tout $1 \leq \beta \leq \alpha$, il existe un sous-groupe de $G$ d'ordre $p ^{\beta}$.
\end{thm}

\begin{thm}[Deuxième théorème de Sylow]
  Le nombre $n_p$ de $p$-Sylow de $G$ vérifie :
  \[
  \begin{cases}
    n_p \equiv 1 \mod p \\
    n_p \mid q.
  \end{cases}
  \]
\end{thm}

\begin{thm}[Troisième théorème de Sylow]

  \

  \begin{enumerate}
    \item Le conjugué d'un $p$-Sylow est un $p$-Sylow.
    \item Tous les $p$-Sylow sont conjugués entre eux.
  \end{enumerate}
\end{thm}

\begin{exo}
  Montrer qu'il n'existe pas de groupes simples d'ordre 15.
\end{exo}

\begin{proof}
  Soit $G$ un groupe d'ordre $3 \times 5 = 15$. D'après le premier théorème de Sylow, $G$ admet au moins un 3-Sylow.

  Soit $n_3$ le nombre de 3-Sylow de G. Par le deuxième théorème de Sylow, on a

  \[
  n_3 \equiv 1 \mod 3 \text{ et } n_3 \mid 5.
  \]

  $G$ admet donc un unique 3-Sylow $H$.

  D'après le (1) du troisième théorème de Sylow, les conjugués de $H$ sont des 3-Sylow de $G$, donc sont égaux à $H$ puisque c'est le seul 3-Sylow de $G$. Donc $H$ est égal à tous ses conjugués et donc $H$ est distingué dans $G$. Puisque $\lvert H \rvert = 3$, $H \neq \{ e \} $ et $H \neq G$. Donc $G$ admet un sous-groupe distingué propre. Donc $G$ n'est pas simple.
\end{proof}

\subsection{Groupes agissant sur un ensemble ou action de groupes}

\begin{definition}[Action de groupe]
  Une action (à gauche) d'un groupe $G$ sur un ensemble $X$ est une application

  \[
    \begin{array}{lll}
    G \times X & \longrightarrow & X \\
    (g,x) & \longmapsto g \cdot x
    \end{array}
  \]

  telle que

  \begin{enumerate}
    \item $\forall x \in X, e \cdot x = x$ (où $e$ est l'élément neutre de $G$) ;
    \item $\forall g, g' \in G, \forall x \in X, g \cdot (g' \cdot x) = (gg')\cdot x$.
  \end{enumerate}
\end{definition}

On peut voir une action comme un morphisme de groupes de $G$ dans le groupe symétrique $\mathfrak{S}_{X} $ de permutations dans $X$ :

\begin{prop}
  Si un groupe $G$ agit sur un ensemble $X$ par

  \[
    \begin{array}{rcl}
    G \times X & \longrightarrow & X \\
    (g,x) & \longmapsto g \cdot x,
    \end{array}
  \]

  alors pour tout $g \in G$, l'application

  \[
  \pi_g:
    \begin{array}{rcl}
    X & \longrightarrow & X \\
    x & \longmapsto g \cdot x
    \end{array}
  \]

  est une permutation de $X$ et l'application

  \[
  \pi:
    \begin{array}{rcl}
    G & \longrightarrow & \mathfrak{S}_{X}  \\
    g & \longmapsto \pi_g
    \end{array}
  \]

  est un morphisme de groupes.

  Réciproquement, si $
    \begin{array}{rcl}
    G & \longrightarrow & \mathfrak{S}_{X}  \\
    g & \longmapsto p_g
    \end{array}$ est un morphisme de groupes, alors $(g,x) \mapsto g \cdot x :=p_g(x)$ est une action de $G$ sur $X$.
\end{prop}

\begin{proof}

  \

  %Vérifions que $\pi$ est bien un morphisme de groupes. Montrons tout d'abord que, si $g \in G$, alors l'application

  %\[
  %\pi:
  %  \begin{array}{rcl}
  %  X & \longrightarrow & X \\
  %  x & \longmapsto \pi_g(x) = g \cdot x
  %  \end{array}
  %\]

  %est une permutation de $X$.

  %On a

  %\[
  %\pi_e:
  %  \begin{array}{rcl}
  %  X & \longrightarrow & X \\
  %  x & \longmapsto x
  %  \end{array} = id_X.
  %\]

  %De plus, si $g, g' \in G$, alors $\pi _{gg'} = \pi_g \circ \pi _{g'}$.

  %Soit $x \in X$, on a $\pi _{gg'}(x) = (gg') \cdot x$ et

  %\begin{gather*}
  %  (\pi_g \circ \pi _{g'})(x)= \pi_g(\pi _{g'})(x)
  %\end{gather*}

  Supposons que $G$ agisse sur un ensemble $X$ par $
    \begin{array}{rcl}
    G \times X & \longrightarrow & X \\
    (g,x) & \longmapsto g \cdot x
    \end{array}$.

  Soit $g \in G$. Considérons l'application $\pi_g:
    \begin{array}{rcl}
    X & \longrightarrow & X \\
    x & \longmapsto g \cdot x
    \end{array}$.

  Montrons que $\pi_g$ est injective. Soient $x, y \in X$ tq $\pi_g(x) = \pi_g(y)$. D'où $g \cdot x = g \cdot y$. D'où $g ^{-1}  \cdot g \cdot x = g ^{-1}  \cdot g \cdot y$. D'où $(g ^{-1}  g ) \cdot x = (g ^{-1}  g ) \cdot y$. D'où $e \cdot x = e \cdot y$. Donc $\pi_g$ est injective. Montrons que $\pi_g$ est surjective. Soit $y \in X$. On a $y = \pi_g(g ^{-1}  y) = g \cdot g ^{-1}  \cdot y$. Donc $\pi_g$ est surjective. Donc $\pi_g$ est bijective.

  On peut donc considérer l'application $\pi:
    \begin{array}{rcl}
    G & \longrightarrow & \mathfrak{S}_{X}  \\
    g & \longmapsto \pi_g
    \end{array}$.

  Montrons que $\pi$ est un morphisme de groupes. Montrons que $\forall g, g' \in G, \pi _{g g'} = \pi_g \circ \pi _{g'}$.

  Soient $g, g' \in G$. Soit $x \in X$. $$\pi _{ gg'}(x) = (g g') \cdot x = g \cdot g' \cdot x = g \cdot (\pi _{ g'} (x)) = \pi_g (\pi _{ g'}(x)).$$

  Donc $\pi _{ gg'} = \pi_g \circ \pi _{ g'}$.

  \

  Réciproquement, si on se donne un morphisme de groupes d'un groupe $G$ dans un groupe de permutations $\mathfrak{S}_{X} $ :

  \[
  p:
    \begin{array}{rcl}
    G & \longrightarrow & \mathfrak{S}_{X}  \\
    g & \longmapsto p_g,
    \end{array}
  \]

  alors l'application

  \[
    \begin{array}{rcl}
    G \times X & \longrightarrow & X \\
    (g,x) & \longmapsto g \cdot := p_g(x)
    \end{array}
  \]

  est une action de groupes.

  En effet,

  \begin{enumerate}
    \item Soit $x \in X$, on a $e \cdot x = p_e(x) = id_X(x) = x$, car $p$ est un morphisme de groupes et l'image de l'élément neutre par un morphisme de groupes est l'élément neutre.
    \item Soient $g, g' \in G$ et soit $x \in X$ ; on a

    \begin{gather*}
      g \cdot (g' \cdot x) = g \cdot (p _{g'}(x)) = p_g(p _{g'}(x)) = (p_g \circ p _{g'})(x) = p _{gg'}(x) = (gg') \cdot x,
    \end{gather*}

    car $p$ est un morphisme de groupes.
  \end{enumerate}
\end{proof}

Cela établit deux bijections réciproques entre l'ensemble des actions de $G$ sur $X$ et celui des morphismes de $G$ dans $\mathfrak{S}_{X} $.

\begin{definition}
  Si un groupe $G$ agit sur un ensemble $X$, alors la relation sur $X$ définie par : pour $x, y \in X, x \sim y $ ssi $\exists g \in G, y = g  \cdot x$ est une relation d'équivalence. La classe d'équivalence de $X$ pour cette relation s'appelle \textbf{l'orbite} de $X$ :

  \[
  Orb(x):= \{ g \cdot x, g \in G \} .
  \]

  Ainsi, l'ensemble des orbites forme une \textbf{partition} de $X$.

  On dit que $g$ agit \textbf{transitivement} s'il n'y a qu'une seule orbite.

  Le \textbf{noyau} de l'action est le noyau du morphisme associé :

  \[
  \pi:
    \begin{array}{rcl}
    G & \longrightarrow & \mathfrak{S}_{X}  \\
    g & \longmapsto \left(\pi_g:
      \begin{array}{rcl}
      X & \longrightarrow & X \\
      x & \longmapsto g \cdot x
      \end{array}\right)
    \end{array}
  \]

  \[
  \operatorname{Ker}(\pi) = \{ g \in G \mid \forall x \in X, g \cdot x = x \}.
  \]

  On dit que l'action est \textbf{fidèle} si son noyau est réduit à $\{ e \} $ (i. e. si le morphisme $\pi$ est injectif).

  Le \textbf{stabilisateur} (ou groupe d'isotropie) d'un élément $x \in X$ est l'ensemble :

  \[
  Stab(x) = \{ g \in G \mid g \cdot x = x \}.
  \]

  C'est un sous-groupe de $G$ (en exercice).
\end{definition}

\begin{prop}
  Pour $x$ fixé dans $X$, l'application

  \[
    \begin{array}{rcl}
    G & \longrightarrow & X \\
    g & \longmapsto g \cdot x
    \end{array}
  \]

  définit une bijection de l'ensemble $G / Stab(x)$ des classes à gauche modulo $Stab(x)$ sur l'orbite de $x$. Ainsi, le cardinal de l'orbite $Orb(x)$ est égal à l'indice du stabilisateur de $x$ :

  \[
  \sharp(Orb(x)) = [G:Stab(x)].
  \]
\end{prop}

\begin{thm}[Formule des classes]
  Soit $G$ un groupe fini agissant sur un ensemble fini $X$. Alors

  \[
  \sharp(X) = \sum_{\substack{x \text{ décrivant un système} \\ \text{des représentants des orbites} }}^{} [G:Stab(x)].
  \]
\end{thm}

\begin{proof}
  \[
  \sharp(X) = \sum_{i=1}^{m} \sharp(Orb(x_i)),
  \]

  où $\{ x_1, \dots, x_n \}$ est un système des représentants des orbites pour l'action de $G$ sur $X$.
\end{proof}

\paragraph{Exemple d'action de groupe}

 On fait agit un groupe $G$ sur lui-même par conjugaison

 \[
   \begin{array}{rcl}
   G \times G & \longrightarrow & G \\
   (g,x) & \longmapsto g \cdot x := g x g ^{-1} .
   \end{array}
 \]

C'est bien une action de groupes, car

\begin{enumerate}
  \item Soit $x \in G$, on a $e \cdot x = e x e ^{-1} =x$.
  \item Soient $g, g' \in G$ et $x \in G$. On a :

  \[
  g \cdot (g' \cdot x) = g \cdot (g x g ^{-1} )=g(g'x (g')^{-1} ) g ^{-1} = (gg') x ((g')^{-1} g ^{-1} ) = (gg')x(gg') ^{-1} = (gg')\cdot x.
  \]
\end{enumerate}

\emph{Cette action est-elle transitive, fidèle ? Quelle est l'orbite d'un élément ?}

\marginpar{20-09-2023}

Soit $x \in G$. L'orbite de $x$ est :

\[
Orb(x) = \{ g \cdot x, g \in G \} = \{ g x g ^{-1} , g \in G \} = \text{classe de conjugaison de } x \text{ dans } G.
\]

On a $Orb(e) = \{ e \} $. Si $G$ n'est pas réduit à $\{ e \} $, il y a plusieurs orbites : l'action n'est donc pas transitive (il y a autant d'orbites que de classes de conjugaison).

\emph{L'action est-elle fidèle ?}  Etudions le noyau du morphisme $\pi$ associé à cette action

\[
\pi:
  \begin{array}{rcl}
  G & \longrightarrow & \mathfrak{S}_{G}  \\
  g & \longmapsto \left( \pi_g:
    \begin{array}{rcl}
    G & \longrightarrow & G \\
    x & \longmapsto g x g ^{-1}
    \end{array}\right).
  \end{array}
\]

On a

\begin{gather*}
  \operatorname{Ker}(\pi) = \{  g \in G \mid \pi_g =id_G \} = \{ g \in G \mid \forall x \in G, \pi_g(x) = x \} \\
  = \{ g \in G \mid \forall x \in G, g x g ^{-1} =x \} =  \{ g \in G \mid \forall x \in G, gx = xg \} = Z(G).
\end{gather*}

L'action est fidèle si et seulement si le centre de $G$ est réduit à l'élément neutre.

Soit $x \in G$. \emph{Quel est le stabilisateur de $x$ ?}

\begin{gather*}
  Stab(x) = \{ g \in G \mid g \cdot x = x \} = \{ g \in G \mid g x g ^{-1} =x \} = \{ g \in G \mid gx=xg \} = \text{centralisateur de } x.
\end{gather*}

Etudions un exemple avec $G = \mathfrak{S}_{3} $. Les orbites de $\mathfrak{S}_{3} $ pour cette action sont les classes de conjugaison de $\mathfrak{S}_{3} $. Elles constituent une partition de $\mathfrak{S}_{3} $.

\begin{enumerate}
  \item $Orb(e) = \{ e \} $.
  \item $Orb(\tau_3) = \{ \sigma \tau_3 \sigma ^{-1} , \sigma \in \mathfrak{S}_{3}  \} = \{ \text{transpositions de } \mathfrak{S}_{3} \} = \{ \tau_1, \tau_2, \tau_3 \} $.
  \item $Orb(\sigma_1) = \{ \sigma \sigma_1 \sigma ^{-1} , \sigma \in \mathfrak{S}_{3}  \} = \{ 3-\text{cycles de } \mathfrak{S}_{3}  \} = \{ \sigma_1, \sigma_2 \} $.
\end{enumerate}

La formule des classes s'écrit alors :

\[
\lvert \mathfrak{S}_{3} \rvert = \sum_{}^{} [\mathfrak{S}_{3}: Stab(x_i)],
\]

où $\{ x_1, x_2, x_3 \} $ est un système des représentants de l'orbite, avec $x_1 = e, x_2 = \tau_1, x_3 = \sigma_1$.

On a

\[
\lvert \mathfrak{S}_{3}  \rvert = \sum_{i=1}^{3} \sharp Orb(x_i)  =  \sharp Orb(x_1) +\sharp Orb(x_2) + \sharp Orb(x_3) =1+3+2 = 6.
\]

L'action est fidèle, car $Z(\mathfrak{S}_{3} ) = \{ e \} $. L'action n'est pas transitive, car il y a trois orbites, à savoir les trois classes de conjugaison.

\[
Stab(e) = \{ \sigma \in \mathfrak{S}_{3} \mid \sigma e = e \sigma \}= \mathfrak{S}_{3}.
\]

On a bien

\begin{gather*}
  [\mathfrak{S}_{3}: Stab(e) ] = \frac{\lvert \mathfrak{S}_{3}  \rvert}{\lvert Stab(e) \rvert} = \frac{3!}{3!}=1 = \sharp Orb(e).
\end{gather*}

On a $[\mathfrak{S}_{3}: Stab(\tau_3) ] = \sharp Orb(\tau_3) = 3$, donc $\lvert Stab(\tau_3) \rvert = 2$. D'où

\begin{gather*}
  Stab(\tau_3) = \{ \text{permutations de } \mathfrak{S}_{3} \text{ qui commutent avec } \tau_3 \} = \{ e, \tau_3 \}.
\end{gather*}

On a $[\mathfrak{S}_{3}: Stab(\sigma_1) ] = \sharp Orb(\sigma_1)=2$, donc $\lvert Stab(\sigma_1) \rvert = 3$. Puisque l'indice du stabilisateur est 2, on en déduit que $Stab(\sigma_1) \triangleleft \mathfrak{S}_{3} $. Or les seuls sous-groupes distingués de $\mathfrak{S}_{n} $ sont $\{ e \}, \mathfrak{S}_{n} \text{ et }  \mathfrak{A}_n  $. Donc

\begin{gather*}
  Stab(\sigma_1) = \mathfrak{A}_3 = \{ e, \sigma_1, \sigma_2 \}.
\end{gather*}

\chapter{Représentations linéaires des groupes finis}

Théorie introduite par Frobenius à la fin du XIX siècle.

\section{Premières définitions}

\begin{definition}
  Une représentation linéaire d'un groupe $G$ est la donnée d'un $\mathbb{C}$-espace vectoriel $V$ muni d'une action de groupes (à gauche) de $G$ agissant de manière linéaire :

  \[\begin{matrix}
    G \times V & \longrightarrow & V \\
    (g,x) & \longmapsto & g \cdot x
  \end{matrix}\]
  %  \begin{array}{rcl}
  %  G \times V & \longrightarrow & V \\
  %  (g,x) & \longmapsto g \cdot x
  %  \end{array}

  telle que

  \begin{enumerate}
    \item $\forall x \in V, e \cdot x = e$ ;
    \item $\forall g, g' \in G, \forall x \in V, g \cdot (g' \cdot x) = (gg')\cdot x$ ;
    \item $\forall g \in G, \forall x,x' \in V, \forall \lambda, \lambda' \in \mathbb{C}$, $g \cdot (\lambda x+\lambda'x') = \lambda g \cdot x + \lambda' g \cdot x$.
  \end{enumerate}
\end{definition}

Une représentation linéaire d'un groupe $G$ est donc la donnée d'un $\mathbb{C}$-espace vectoriel $V$ et d'un morphisme de groupes :

%\[
%\rho:
%  \begin{array}{rcl}
%  G & \longrightarrow & GL(V) \\
%  g & \longmapsto \left(\rho_g:
%    \begin{array}{rcl}
%    V & \longrightarrow & V \\
%    x & \longmapsto g \cdot x
%    \end{array}\right),
%  \end{array}
%\]


\[\begin{matrix}
\rho : & G & \longrightarrow & GL(V) \\
\ & g & \longmapsto & \left(\begin{matrix}
 V & \longrightarrow & V \\
 x & \longmapsto & g \cdot x
\end{matrix}\right)
\end{matrix}\]
où $GL(V)$ est le groupe des automorphismes du $\mathbb{C}$-espace vectoriel $V$.

On a bien $\forall g, g' \in G, \rho _{gg'} = \rho_g \circ \rho _{g'}$ et $\rho_e=id_V$ et $\rho _{g ^{-1} } = \rho_g ^{-1} $ comme vu précédemment.

De plus, $, \forall  g \in G, $ la bijection $\rho_g$ est un endomorphisme de $V$, i. e. une application linéaire de $V$ dans $V$ et donc $\rho_g \in GL(V)$. En effet, si $x, x' \in V$ et $\lambda , \lambda' \in \mathbb{C}$, alors

\begin{gather*}
  \rho_g(\lambda x + \lambda' x') = g \cdot(\lambda x+ \lambda'x') \stackrel{(3)}{=} \lambda g \cdot x + \lambda' g \cdot x' = \lambda \rho_g(x) + \lambda'\rho_g(x').
\end{gather*}

\begin{definition}
  L'espace vectoriel $V$ est appelé \textbf{l'espace de la représentation}.

  La dimension de $V$ (en tant que $\mathbb{C}$-espace vectoriel) est appelé le \textbf{degré} ou la dimension de la représentation.

  Lorsque $\rho$ est injectif, la représentation est dite \textbf{fidèle} ; le groupe $G$ se représente alors de manière concrète comme un sous-groupe de $GL(V)$ ; lorsque $V$ est de dimension finie (ce que nous allons supposer dorénavant), le choix d'une base du $\mathbb{C}$-espace vectoriel $V$ fournit alors une représentation encore plus concrète comme groupe de matrice.
\end{definition}

\begin{remark}[Personnelle]
  Si \(\rho\) est une représentation fidèle, alors

  \[\operatorname{Ker}(\rho) = \{ g \in G \mid \forall x \in V, g \cdot x = x \} = \{ e \}. \]
\end{remark}

\begin{remark}
  Soient $G$ un groupe fini et $\rho : G \to GL(V)$ une représentation (linéaire) de $G$. Soit $g \in G$ un élément d'ordre $n$. On a alors

  \[
  (\rho_g)^n = \rho _{g ^{n}} = \rho_e = id_V.
  \]

  Donc l'endomorphisme $\rho_g$ est racine du polynôme $X^{n}-1$ qui n' a que des racines simples. Le polynôme minimal de $\rho_g$ divise donc le polynôme $X ^{n}-1$ et n'a donc aussi que des racines simples. Le polynôme minimal de $\rho_g$ est donc scindé sur $\mathbb{C}$ et à racines simples, on en déduit que l'endomorphisme de $\rho_g$ est \textbf{diagonalisable}.
\end{remark}

\paragraph{Exemples}

\begin{enumerate}
  \item La représentation triviale (ou représentation unité) :

  \[\begin{matrix}
  \rho : & G & \longrightarrow & GL(\mathbb{C}) \simeq \mathbb{C}^{*} \\
  \ & g & \longmapsto & \left( \rho_g : \begin{matrix}
   \mathbb{C} & \longrightarrow & \mathbb{C} \\
   x & \longmapsto & x
  \end{matrix}\right).
  \end{matrix}\]


    \item Les représentations de degré 1 : ce sont les morphismes de groupes

    \[
    \rho : G \longrightarrow \mathbb{C} ^{*}
    \]

    puisque si $dim V =1$, alors $GL(V) \simeq \mathbb{C}^{*}$, car les endomorphismes de $V$ sont des homothéties :

    %\[
    %f _{\lambda} : \mathbb{C} \longrightarrow \mathbb{C}, f _{\lambda}(x) = \lambda x
    %\]

    \[\begin{matrix}
    f _{\lambda } : & \mathbb{C} & \longrightarrow & \mathbb{C} \\
    \ & x & \longmapsto & \lambda x
    \end{matrix}\]

    et

    \[\begin{matrix}
      GL(V) & \longrightarrow & \mathbb{C} ^{*} \\
      f_{\lambda} & \longmapsto & \lambda
    \end{matrix}\]

    %\[
    %  \begin{array}{rcl}
    %  GL(V) & \longrightarrow & \mathbb{C} ^{*} \\
    %  f _{\lambda } & \longmapsto \lambda
    %  \end{array}
    %\]

    qui a une homothétie fait correspondre son rapport induit un isomorphisme. Si $G$ est \textbf{fini}, tout élément de $G$ est d'ordre fini (par le théorème de Lagrange) et donc, pour tout $g \in G$, $\rho_g$ est une racine de l'unité dans $\mathbb{C}$, et en particulier $\rho_g$ est un nombre complexe de module 1 :

    \[
    \lvert \rho_g \rvert = 1.
    \]

    \item Soient $\mathfrak{S}_n$ le groupe symétrique et $(e_1, \dots, e_n)$ la base canonique de $\mathbb{C} ^{n}$. On définit la représentation canonique de degré $n$ de $\mathfrak{S}_n$ en posant :

    \[\begin{matrix}
    \rho : & \mathfrak{S}_n & \longrightarrow & GL(\mathbb{C} ^{n}) \\
    \ & \sigma & \longmapsto & \left( \rho _{\sigma} : \begin{matrix}
    \mathbb{C} ^{n} & \longrightarrow & \mathbb{C}^{n} \\
    e_i & \longmapsto & \rho _{\sigma}(e_i) := e _{\sigma(i)}
    \end{matrix}\right).
    \end{matrix}\]

    %\[
    %\rho:
    %  \begin{array}{rcl}
    %  \mathfrak{S}_n & \longrightarrow & GL(\mathbb{C} ^{n}) \\
    %  \sigma & \longmapsto \left(\rho_\sigma:
    %    \begin{array}{rcl}
    %    \mathbb{C} ^{n} & \longrightarrow & \mathbb{C} ^{n} \\
    %    e_i & \longmapsto \rho _{\sigma}(e_i) := e _{\sigma(i)}
    %    \end{array}\right).
    %  \end{array}
    %\]

\end{enumerate}


\end{document}
