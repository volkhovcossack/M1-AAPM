\documentclass[french]{book}
\usepackage[utf8x]{inputenc}
\usepackage[T1]{fontenc}
\usepackage{babel}
\usepackage{lmodern}
\usepackage[top=2cm,bottom=2cm,left=3cm,right=3cm]{geometry}
\usepackage{microtype}
\usepackage{mathtools, amssymb, amsthm}
\usepackage{mdframed}
\usepackage{hyperref}
\usepackage{graphicx}
\usepackage{xcolor}
\usepackage{mathrsfs}
\usepackage{wrapfig}
\usepackage{stmaryrd}
\usepackage{framed}
\usepackage{dsfont}
\usepackage{marginnote}
\usepackage{svg}
\usepackage[Glenn]{fncychap}

\newtheorem{prototheorem}{Théorème}[section]
\newenvironment{thm}
   {\colorlet{shadecolor}{orange!10}\begin{shaded}\begin{prototheorem}}
   {\end{prototheorem}\end{shaded}}

\newtheorem*{protocorollary}{Corollaire}
\newenvironment{corollary}
    {\colorlet{shadecolor}{violet!10}\begin{shaded}\begin{protocorollary}}
    {\end{protocorollary}\end{shaded}}

\newtheorem*{protolemma}{Lemme}
\newenvironment{lemma}
    {\colorlet{shadecolor}{pink!15}\begin{shaded}\begin{protolemma}}
    {\end{protolemma}\end{shaded}}

\theoremstyle{definition}
\newtheorem{protodefinition}{Définition}[section]
\newenvironment{definition}
    {\colorlet{shadecolor}{green!5}\begin{shaded}\begin{protodefinition}}
    {\end{protodefinition}\end{shaded}}

\newtheorem{protoproposition}{Proposition}[section]
\newenvironment{prop}
    {\colorlet{shadecolor}{blue!5}\begin{shaded}\begin{protoproposition}}
    {\end{protoproposition}\end{shaded}}

\theoremstyle{remark}
\newtheorem*{remark}{Remarque}
\newtheorem{exo}{Exercice}
\newtheorem*{protoexemple}{Exemple}
\newenvironment{exemple}
    {\colorlet{shadecolor}{gray!10}\begin{shaded}\begin{protoexemple}}
    {\end{protoexemple}\end{shaded}}


\newcommand{\lesss}{<}
\newcommand{\less}{\lesss}

\newcommand{\biggg}{>}
\newcommand{\bg}{\biggg}

\renewcommand\qedsymbol{$\clubsuit$}

\title{\bsc{Analyse fonctionnelle et distributions}}
\author{Michel \bsc{Rouleux}}
\date{2023-2024}


\begin{document}

\maketitle

\tableofcontents

\chapter{Espaces localement convexes}

\section{Rappels de topologie}

J. Dieudonné 1 et 2.

Reed-Simon 1, 2 et 4.

Brézis, ``Analyse fonctionnelle''

\

Soit $X$ ensemble. Soit $(X, \mathscr{T} )$ espace topologique où $\mathscr{T}  \subset \mathscr{P}(X) $.

$\mathscr{T} $ parcourt l'ensemble des voisinages de $x$ où $x$ est un point quelconque de $X$.

\subsection{Axiomes}

\begin{enumerate}
  \item Soient $x \in X$ et $ V'$ voisinage de $x$. Si $V \supset V'$ alors $V$ est un voisinage de $X$.
  \item $\bigcap_{\text{ finie } }  V_i$ est un voisinage de $x$, $\bigcap_{\text{ finie } } V_i \in \mathscr{T}  $,  mais $\bigcap_{\varepsilon \bg 0} V _{\varepsilon }  \neq \emptyset$ n'est pas un voisinage de $0$.
  \item $\bigcap_{i \in I} V_i $ est un voisinage de $x$.
\end{enumerate}


\begin{definition}[Ouvert]
  $\Omega$ ouvert si et seulement si $\Omega$ est voisinage de chacun de ses points.
\end{definition}


\begin{exemple}
  $(-1, 1)$ ouvert tandis que $[-1, 1)$ non ouvert car -1 n'a pas de voisinage.
\end{exemple}

\

$V(x) =(x- \varepsilon , x+ \varepsilon )$ est une base de voisinage pour la topologie usuelle de $\mathbb{R}$.

\begin{exo}
  On peut définir axiomatiquement $\mathscr{T} $ à partir de ses ouverts.
\end{exo}

\begin{definition}[Fermé]
  On dit que $F$ est un fermé si et seulement si $F ^{C}$ est un ouvert.
\end{definition}

\subsection{Cas particulier d'espaces topologiques : espaces métriques}



\begin{definition}[Espace métrique, distance] \label{distance}
  \

  $X$ est un ensemble, $d : X \times X \to \mathbb{R} ^{+}$ distance sur $X$ si et seulement si :
  \begin{enumerate}
    \item $d(x, y) = 0$ si et seulement si $x = y$ ;
    \begin{remark}
      Si on a seulement $x =y \implies d(x,y) =0$, alors $d$ est un écart.
    \end{remark}
    \item $d(x,y) = d(y,x)$ (symétrie) ;
    \item $d(x,y) \leq d(x,z) + d(z,y)$ (inégalité triangulaire).
    De ce fait, $\lvert d(x,z) - d(y,z) \rvert \leq d(x,y)$.
  \end{enumerate}
\end{definition}



\begin{exemple}
  \begin{enumerate}
    \item Dans $\mathbb{R}^n$, $d(x,y) = \Vert x-y \Vert $.
    \item $X$ ensemble. On définit $d$ de la façon suivante :
    $$ \forall x, y \in X,  d(x, y) = \begin{cases}
      d(x,y) =0 \text{ si } x=y \\
      d(x,y) = 1 \text{ si } x \neq y.
    \end{cases}$$

    Il s'agit de la distance triviale.

    Si $x,y,z$ distincts alors $d(x,y) \leq d(x,z)+d(z,y)$.
  \end{enumerate}

\end{exemple}


\subsection{Comparaison des topologies}

\

Soient $X$ un ensemble et $\mathscr{T}, \mathscr{T}' $ des topologies sur $X$.

\begin{definition}[Plus fine]
  On dit que $\mathscr{T}' $ est plus fine que $\mathscr{T} $ et on note $\mathscr{T}' \prec \mathscr{T}  $ si et seulement si $\mathscr{T} \subset \mathscr{T}' $.

  On dit aussi que $\mathscr{T}' $ est plus forte que $\mathscr{T} $.
\end{definition}

\begin{remark}
  Si $\mathscr{T}' \prec \mathscr{T}  $, il y a plus d'ouverts dans $\mathscr{T}' $ que dans $\mathscr{T} $ (idem pour les fermés).
\end{remark}

\begin{proof}
  Soit $\Omega$ ouvert dans $X$. On a $\Omega \in \mathscr{T}  \implies \Omega \in \mathscr{T}' $.

  Soit $F$ un fermé dans $X$. On a $F \in \mathscr{T}$, mais $ \Omega = F ^{C} \in \mathscr{T} \implies F ^{C} \in \mathscr{T}' $, donc $F \in \mathscr{T}' $.
\end{proof}

\paragraph{Formulations équivalentes}

\begin{enumerate}
  \item On suppose que $\mathscr{T}' \prec \mathscr{T} $. Si $\forall x \in X $, $U$ est un voisinage de  $x$ pour $\mathscr{T} $, alors $U $ voisinage de $x$ pour $\mathscr{T}' $, car si $U $ est un ouvert de $\mathscr{T} $, alors $U $ est un ouvert de $ \mathscr{T}' $.
  \item Pour l'application identité définie comme suit

  $$ id : (X, \mathscr{T}' ) \longrightarrow (X, \mathscr{T} ), $$

  %allant de $(X, \mathscr{T}')$ vers $(X, \mathscr{T} )$,

  on a $\mathscr{T}' \prec \mathscr{T}  $ si et seulement si $id $ est continue.
\end{enumerate}

\

Par exemple, prenons $X = \{ f :[0, 1] \to \mathbb{R} \} $. On prend $\mathscr{T} $ topologie de la convergence simple, i. e.

$f_n $ converge vers $ f$ simplement si $\forall x \in [0, 1], f_n(x) \to f(x)$.

\subparagraph{Ouverts de $\Omega$}

$$\Omega _{a, \varepsilon } = \{ f \in X \mid \sup _{i =1, \dots, k} \lvert f(a_i) \rvert \less \varepsilon \}, $$ avec $a = a_0, \dots, a_k \in [0, 1]$ et $\varepsilon  \bg 0$.

$\Omega _{a, \varepsilon } $ est un voisinage de 0 (la fonction nulle) dans $X$.

Pour $f_0 \in X$, $\Omega _{a, \varepsilon } + f_0$ est une base de voisinage de $f_0$, car $X$ est un espace vectoriel (on agit par translation).

\

On considère maintenant la topologie de la convergence uniforme $\mathscr{T}' $.

$\Omega _{\varepsilon } = \{ f \in X, \sup_{ x \in [0, 1] } \lvert f(x) \rvert \less \varepsilon   \} $ est un voisinage de $0$ (la fonction nulle).

\begin{prop}
  $\mathscr{T}' $ est plus fine que $\mathscr{T} $, ie $\mathscr{T} \subset \mathscr{T}' $.
\end{prop}

\begin{proof}
  Soit $\Omega _{a, \varepsilon } \in \mathscr{T}  $.

%  $f \in \Omega _{a, \varepsilon }$, alors on a $\lvert f(a_i) \rvert \less \varepsilon $.

  Si $f \in \Omega _{\varepsilon }$, alors $$\forall x \in [0, 1], \lvert f(x) \rvert \less \varepsilon, $$ ce qui implique que $$ \forall i \in \{ 1, \dots, k \}, \lvert f(a_i) \rvert \less \varepsilon \ (\text{car c'est vrai pour tout } x). $$

   Donc $\Omega  _{\varepsilon } $ est un voisinage de 0 dans $\mathscr{T} $. On a ainsi démontré que $\mathscr{T}' $ est plus fine que $\mathscr{T} $.

  %Or $\Omega _{a, \varepsilon } \subset \Omega _{\varepsilon }$, donc $\Omega _{ \varepsilon }$ est un voisinage de $0$ pour la topologie $\mathscr{T} $.
\end{proof}

\

On considère l'espace des fonctions continues $\mathcal{C}^0$ avec la norme $$ \Vert f \Vert _{0} = \sup_{  } \lvert f(x) \rvert $$ et l'espace des fonctions de classe $\mathcal{C}^1$ $\mathcal{C}^1$ avec la norme $$\Vert f \Vert _{1} = \sup \lvert f(x) \rvert + \sup \lvert f'(x) \rvert.$$

La topologie sur $\mathcal{C}^1$ est plus fine que celle sur $\mathcal{C}^0$.

\begin{proof}
  On a pour tout $f$, $$\Vert f \Vert _{0} \leq \Vert f \Vert _{1} .$$

  Ainsi si $$\Vert f \Vert _{1} \less \varepsilon,   $$ alors $$\Vert f \Vert _{0} \less \varepsilon.  $$

  Par conséquent, $\{ f , \Vert f \Vert _{1 } \less \varepsilon  \} \subset \{ f, \Vert f \Vert _{0} \less \varepsilon   \} $.

  Donc $\mathscr{T}' \prec \mathscr{T}  $.

  On sait également que si $U$ est un voisinage de 0 pour $\mathscr{T} $, alors $U$ est un voisinage de 0 pour $\mathscr{T}' $.
\end{proof}

\paragraph{Topologie métrisable (exemples)}

\begin{enumerate}
  \item Topologie grossière $\mathscr{T}= \{ \emptyset, X \} $. C'est la topologie la moins fine.

  \begin{remark}
    $\mathscr{T}' = \mathscr{P}(X)  $ est la topologie la plus fine.
  \end{remark}

  Vérifions si la topologie grossière est métrisable dans différents cas.
  \begin{itemize}
    \item Si $X = \{ a \} $, on a $d(a,a) =0$. Le seul voisinage de $a$ est $X = \{ a \} $. Donc $\mathscr{T} $ est métrisable.
    \item Supposons que $X = \{ a,b \} $. Mais $\mathscr{T} $ n'est plus métrisable, avec $d(a,b) =1$ (distance triviale).

    Raisonnons par l'absurde. Si $\mathscr{T} $ était métrisable, $\mathscr{T} $ devrait contenir un ouvert $\Omega$ tel que $a \in \Omega$ et $b \notin \Omega$. Or $\mathscr{T} = \{ \emptyset, X \} $, donc c'est impossible.
  \end{itemize}

  Pour $\mathscr{T}' $, on choisit la distance $d$ telle que $d(x,y) = 0 \text{ ou } 1$. Est-ce que $\mathscr{T}' $ est métrisable ?

  \item Prenons $\mathscr{T} $ telle que $\mathscr{T} = \{ \emptyset, \{ a \} , X \} $.

  On suppose que $X$ contient au moins deux éléments. Dans ce cas, $\mathscr{T} $ est une topologie sur $X$ non métrisable, car si $d(a,b)=1,$ avec $ b \neq a$, alors dans $\mathscr{T} $ il n'existe pas de boule ouverte qui contient $\{ b \} $ sans contenir $\{ a \} $.

  \item Considérons $X = \{ a,b \}$ muni de la topologie $  \mathscr{T} = \{ \emptyset, \{ a \}, \{ b \}, X \} = \mathscr{P}(X)  $.

  On a $d(a,b) = 1$, car $a \neq b$.

  De ce fait :

  $\{ a \} $ voisinage de $a$ qui ne contient pas $b$ ( $\{ a \} = \{ x \text{ tel que } d(x,a) \less 1 \} $) ;

  $\{ b \} $ voisinage de $b$ qui ne contient pas $a$.




\end{enumerate}

\subsection{Espaces vectoriels topologiques}

Dans le cas où $(X, \mathscr{T} )$ est un espace vectoriel topologique, il suffit de connaître les voisinages de 0 et on agit par translation pour déterminer les voisinages de n'importe quel $x \in X$.

\begin{definition}[Continuité]
  Soient $X, Y$ deux espaces vectoriels topologiques et $f: X \to Y$ une application. On considère :

  \begin{gather*}
    (U_a) _{a \in A} \text{ voisinage de } 0 \text{ dans } X \\
    (V_b) _{b \in B} \text{ voisinage de } 0 \text{ dans  } Y
  \end{gather*}

  $f$ est continue si pour tout $V = V_b + f(x_0)$ dans $Y$, il existe $U = \bigcap_{\text{finie} } (U_a +x_0) $ voisinage de $x$ dans $X$ tel que $x \in U \implies f(x) \in V$.
\end{definition}

\paragraph{Cas particulier : $X$ normé}

\begin{definition}[Norme]
  $\Vert \cdot \Vert $ est une norme sur $X$ si

  \begin{enumerate}
    \item $\Vert x \Vert = 0 \iff x=0 $ (séparation);
    \item $\Vert \lambda x \Vert = \lvert l \rvert \Vert x \Vert $ (absolue homogénéité);
    \item $\Vert x+y \Vert \leq \Vert x \Vert + \Vert y \Vert $ (inégalité triangulaire).
  \end{enumerate}
\end{definition}

De cette norme, on construit la distance $d$ telle que $$\forall x, y \in X, d(x,y) = \Vert x-y \Vert.$$

Voisinages de 0.

$(U_a) = B(0, a)$

$A = \mathbb{R} ^{+}$.

\begin{itemize}
  \item $f: X \to Y$ continue en $x_0$, $\forall V = V_b + f(x_0)$, $\exists U = B(0, \delta ) + x_0 , f(U) \subset V$.
  \item $X, Y$ EVN.

  $\forall \varepsilon  \bg 0, \exists \delta  \bg 0, f(B(0, \delta ) + x_0) \subset B(f(x_0), \varepsilon )$.
\end{itemize}

\marginnote{15-09-2023}

\section{Semi-normes et espaces localement convexes}

\subsection{Semi-normes sur $X$ espace vectoriel}

\begin{definition}[Semi-norme]
  L'application $\rho : X \to \mathbb{R} ^{+}$ est une semi-norme si :
  \begin{enumerate}
    \item $\rho(0) =0$ ;
    \item $\rho(\lambda x) = \lvert \lambda  \rvert \rho(x)$ ;
    \item $\rho(x+y) \leq \rho(x) + \rho(y)$.
  \end{enumerate}
\end{definition}

$X$ est un espace vectoriel $\mathbb{R}$ ou $\mathbb{C}$.

\begin{remark}
  {\fontencoding{U}\fontfamily{futs}\selectfont\char 66\relax} \ On n'a pas forcément $\rho(x) = 0 \implies x=0$.

\end{remark}


\begin{exemple}
  \begin{enumerate}
    \item Si $\rho$ est une norme, c'est aussi une semi-norme.
    \item $X = \mathcal{C}^0([0, 1], \mathbb{R} \ (\text{ou } \mathbb{C}))$. On prend $a = (a_0, \dots, a_k) \subset [0, 1]$. On définit

    \begin{equation*}
      \rho_a (f) = \sup_{ 0 \leq i \leq k } \lvert f(a_i) \rvert .
    \end{equation*}

    \item Topologie faible. $X$ est un espace vectoriel et $X'$ est son dual (espace contenant les formes linéaires sur $X$).

    Soit $l$ une forme linéaire dans $X'$. Alors

    \begin{equation*}
      p(x) = \lvert \langle l,x \rangle  \rvert.
    \end{equation*}
  \end{enumerate}
\end{exemple}



\begin{definition}[Famille de semi-normes séparée]
  Soit $(\rho_a) _{a \in A}$ une famille de semi-normes. On dit que $(\rho_a) _{a \in A}$ sépare les points (ou est séparée) si et seulement si

  \begin{equation*}
    \forall a \in A, \rho_a(x) = 0 \implies x=0.
  \end{equation*}
\end{definition}

\begin{definition}[Espace localement convexe (ELC)]
  L'espace vectoriel topologique $X$ est un \textbf{espace localement convexe} si et seulement si $X$ est \textbf{muni d'une famille de semi-normes qui séparent les points}.
\end{definition}

\begin{prop}
  Si $X$ est un espace localement convexe, alors $X$ est un espace vectoriel topologique pour la topologie définie par $\rho_a$.
\end{prop}

%\begin{remark}[Personnelle, n'est pas donnée dans le cours]
%  Un espace vectoriel topologique est dit localement convexe si
%\end{remark}

\begin{proof}
  On note $\mathscr{T} $ la topologie définie par la famille de semi-normes $(\rho_a) _{a \in A}$.

  \begin{remark}[Personnelle]
    On cherche à montrer que les $\mathcal{O} _{a, \varepsilon }$ forment une topologie. On va vérifier les axiomes de topologie.
  \end{remark}

  Dans ce cas, les ouverts $\mathcal{O} \in \mathscr{T} $ sont $\mathcal{O} = \mathcal{O} _{a, \varepsilon }, a \in A, \varepsilon  \bg 0$ définis ci-dessous :

  \begin{equation*}
    \mathcal{O} _{a, \varepsilon } = \{ x \mid \rho_a(x) \less \varepsilon  \}
  \end{equation*}

  $\mathcal{O} _{a,\varepsilon }$ une base de voisinages de 0.

  Les voisinages de $x$ sont donnés par translation :

  \begin{equation*}
    x + \mathcal{O} _{a, \varepsilon } = \{ x+y, y \in \mathcal{O} _{a, \varepsilon } \}.
  \end{equation*}

  On montre facilement que $\bigcap_{\text{finie} } \mathcal{O} _{a, \varepsilon } \in \mathscr{T} $ et $\bigcup_{\text{quelconque} } \mathcal{O} _{a, \varepsilon } \in \mathscr{T}  $.
\end{proof}

\begin{prop}
  $\mathscr{T} $ est la topologie la moins fine sur $X$ qui rend continues

  \begin{gather*}
    (x,y) \mapsto x+y \text{ et } (\lambda , x) \to \lambda x.
  \end{gather*}

  Il y a donc une compatibilité avec la structure des espaces vectoriels.
\end{prop}

\begin{proof}

  \begin{enumerate}
    \item   $\mathscr{T} $ rend continues les deux opérations de $X$. On a en effet

      \begin{gather*}
        \rho_a(x+y) \leq \rho_a(x)+ \rho_a(y).
      \end{gather*}

      Il suffit de prendre $\rho_a(x) \less \frac{\varepsilon }{2}$ et $\rho_a(x) \less \frac{\varepsilon }{2}$, on obtient $\rho_a(x+y) \less \varepsilon $.

      On a $\rho(\lambda x) = \lvert \lambda \rvert \rho(x)$ et on démontre ce résultat par analogie.

      \item La moins fine (en exercice).
  \end{enumerate}

\end{proof}

\begin{thm}\label{Hausdorff}
  La topologie de $X$ espace localement convexe est Hausdorff, i. e. elle sépare les points.
\end{thm}

\begin{definition}[Hausdorff]
  $(X, \mathscr{T} )$ est de Hausdorff si et seulement si pour tout $x, y \in X$ tel que $x \neq y$, il existe $\mathcal{O}_x$ et $\mathcal{O}_y$ voisinages de $x$ et de $y$ tels que $$ \mathcal{O}_x \cap \mathcal{O}_y = \emptyset.$$
  % tel que $y \notin \mathcal{O}_x$ et il existe $\mathcal{O}_y$ tel que $x \notin \mathcal{O}_y$.
\end{definition}


\begin{exemple}
  On prend $X = \{ a,b \}, \mathscr{T} = \{ \emptyset, \{ a \}, \{ b \}, X \}  $. On a $\{ a \} \cap \{ b \} = \emptyset $. Donc $(X, \mathscr{T} )$ est séparée.
\end{exemple}

\begin{proof}[Démonstration du théorème \ref{Hausdorff}]
  Par contraposée, on prend $y \neq 0$ et $ x = 0$.

  Si $X$ est un espace localement convexe, alors il existe $a \in A$ tel que $\rho_a(y) = \varepsilon \bg 0$.

  On pose
  \begin{gather}
    V_x = \left\{ z, \rho_a(z) \less \frac{\varepsilon }{2} \right\} \text{ et } V_y = \left\{ z, \rho(z-y) \less \frac{\varepsilon }{2} \right\} .
  \end{gather}

  Par l'inégalité triangulaire, on obtient $V_x \cap V_y = \emptyset$, car

  \begin{gather*}
    \rho_a(x-y) \bg \lvert \rho_a(x) - \rho_a(y) \rvert \bg \left\lvert \frac{\varepsilon }{2} - \varepsilon  \right\rvert = \frac{\varepsilon }{2} \bg 0.
  \end{gather*}

  %\begin{equation*}
  %  \rho_a(x-y) \bg
  %\end{equation*}

  %On a $y \in \mathcal{O} _{a, \varepsilon } + y$.

  %On pose $U_0 = \{ x, \rho_a(x) \less \frac{\varepsilon }{2}\} $.

  %Montrons que $x \notin \mathcal{O} _{a, \varepsilon } + y$.

  %On a

  %\begin{equation*}
    %\rho_a(x-y) \bg \varepsilon  - \frac{\varepsilon }{2} - \frac{\varepsilon }{2}.
  %\end{equation*}
\end{proof}

\section{Pourquoi ``localement convexe''?}

\begin{definition}
  Soit $X$ un $\mathbb{R}$ ou $\mathbb{C}$ espace vectoriel.

  \begin{enumerate}
    \item On dit que $C \subset X$ est convexe si

    \begin{equation*}
      \forall x, y \in C, \forall t \in  [0, 1], z = tx + (1-t)y \in C.
    \end{equation*}

    \item On dit que $B \subset X$ est balancé (sur $\mathbb{R}$) ou cerclé (sur $\mathbb{C}$) si

    \begin{equation*}
      \forall \lambda  \in \mathbb{R}, \lvert \lambda  \rvert = 1 \implies \forall x \in B, \lambda x \in B.
    \end{equation*}

    \item On dit que $E \subset X$ est équilibré si

    \begin{equation*}
      \forall \lambda  \in \mathbb{R} \text{ ou } \mathbb{C}, \lvert \lambda  \rvert \leq 1 \implies \forall x \in E, \lambda x \in E.
    \end{equation*}

    \item On dit que $A$ est absorbant si

    \begin{equation*}
      \bigcup _{t \bg 0} t A = X.
    \end{equation*}
  \end{enumerate}
\end{definition}

\begin{figure}[h!]
  \centering
  \includegraphics[scale=0.5]{figures/convexe1.png}
  \caption{Ensemble convexe}
  \label{}
\end{figure}



\begin{exemple}
  \begin{enumerate}
    \item Si $X$ est un espace vectoriel normé, $A = B(0, 1)$ et $x \in X$, on a $\frac{x}{\Vert x \Vert } \in B(0, 1)$. Alors $x \in \Vert x \Vert B(0, 1)$.
    \item Si $0 \in C$ convexe, alors $C$ est équilibré si et seulement si $C$ est balancé.
  \end{enumerate}
\end{exemple}


\begin{proof}
  On suppose que $C$ est balancé. Pour $x \in C \implies -x \in C$, donc $[-x, x] \in C$ par convexité.
\end{proof}

\begin{thm}\label{convexes-balances}
  Soit $X$ un espace vectoriel topologique. Les assertions suivantes sont équivalentes :

  \begin{enumerate}
    \item $X$ est un espace localement convexe (réel ou complexe);
    \item Il existe une base de voisinages de $0 \in X$ qui sont convexes, balancés (cerclés), absorbants.
  \end{enumerate}
\end{thm}

\marginnote{19-09-2023}

\begin{proof}
  \begin{enumerate}
    \item Si $X$ est un espace localement convexe, alors une base de voisinages de 0 est donnée par

    \[
    \mathcal{O} _{a, \varepsilon } = \{ x  \in X \mid \rho_a(x) \less \varepsilon  \}
    \]

    Les $\mathcal{O} _{a, \varepsilon }$ sont convexes, balancés et absorbants (TD).

    \item On utilise la jauge de Minkowski \ref{jauge_mink}.

    On pose \[
    \rho_C(x) = \mu_C(x).
    \]

    et on vérifie que $\rho_C$ est une semi-norme. Grâce au lemme \ref{lemme_mink}, on obtient les résultats suivants :

    \begin{enumerate}
      \item $\rho_C(x+y) \leq \rho_C(x)+ \rho_C(y)$, car $C$ est convexe ;
      \item $\rho_C(\lambda x) = \lambda \rho_C(x)$ si $\lambda \bg 0$ et $\rho_C(\lambda x) = \lvert \lambda  \rvert \rho_C(x)$, car $C$ est cerclé.

       %$\rho_C(\lambda x) = \rho_C((-\lambda )x)$ et on a

      %\[
      %\rho_C(\lambda x) = (-\lambda ) \rho_C(x) = \lvert \lambda  \rvert \rho_C(x),
      %\]

      %car $C$ est cerclé.

      $X$ muni de $\rho_C$ est un espace localement convexe.
    \end{enumerate}
  \end{enumerate}
\end{proof}

\begin{definition}[Jauge de Minkowski] \label{jauge_mink}
  Soit $X$ espace vectoriel réel ou complexe. On suppose que $C$ tel que $0 \in C$ est absorbant. Alors la jauge de Minkowski est définie comme suit :

  \[
  \mu_C(x) = \inf \{ \alpha \bg 0, x \in \alpha C \}.
  \]
\end{definition}

\begin{figure}[h!]
  \centering
  \includegraphics[scale=0.3]{figures/jaugem.png}
  \caption{La jauge de Minkowski}
  \label{}
\end{figure}

\begin{remark}
  Si $C$ est absorbant, alors $\forall x \in X$, $\mu_C(x) \less \infty$.
\end{remark}

\begin{lemma} \label{lemme_mink}
  Soit $C \subset X$ absorbant tel que $0 \in C$.
  \begin{enumerate}
    \item Si $\lambda  \geq 0$, $\mu_C(\lambda x)= \lambda \mu_C(x)$ ;
    \item Si $C$ est convexe, alors $\mu_C(x+y) \leq \mu_C(x)+ \mu_C(y)$ ;
    \item Si $C$ est cerclé, alors $\mu_C(\lambda x) = \lvert \lambda  \rvert \mu_C(x)$ ;
    \item $\{ x \in X, \mu_C(x) \less 1 \} \subset C \subset \{ x \in X, \mu_C(x) \leq 1 \} $.
  \end{enumerate}
\end{lemma}

\subsection{Théorème de Hahn-Banach}

Il y a la forme analytique et la forme géométrique de ce théorème.

\begin{thm}[De Hahn-Banach, forme analytique]
  Pour simplifier, on prend $X$ espace vectoriel sur $\mathbb{R}$. Soit $p : X \longrightarrow \mathbb{R}$ qui vérifie :
  \begin{itemize}
    \item [$\star$] $\forall x \in X, \forall \lambda \bg 0, p(\lambda x) = \lambda p(x)$ ;
    \item [$\star$] $\forall x, y \in X$, $p(x+y) \leq p(x) +p(y)$.
  \end{itemize}

  Soient $Y$ un sous espace vectoriel de $X$ et $l$ une forme linéaire sur $Y$ qui vérifie

  \[
  \forall x \in Y, l(x) \leq p(x), \forall x \in Y.
  \]

  Alors (prolongement) il existe $L$ forme linéaire sur $X$ telle que $L _{|Y} = l$ \textbf{et}

  \[
  \forall x \in X, L(x) \leq p(x).
  \]
\end{thm}

On l'applique aux espaces vectoriels normés, espaces localement convexes,...

\begin{thm}[Norme sur un espace dual] \label{norm-dual}
  Soit $X$ espace vectoriel normé, $X'$ formes linéaires continues sur $X$, $X'$ est un espace vectoriel normé. La norme sur $X'$ est définie de la façon suivante :

  \[
  \Vert L \Vert _{X'} \stackrel{\text{déf}}{=} \sup_{ \substack{x \in X \\ \Vert x \Vert =1 } } \lvert \langle L,x \rangle  \rvert = \sup_{ x \in X \setminus \{ 0 \} } \left\lvert \left\langle L, \frac{x}{\Vert x \Vert } \right\rangle  \right\rvert = \sup_{ x \in X \setminus \{ 0 \} } \frac{\lvert \langle L,x \rangle  \rvert}{\Vert x \Vert }.
  \]

\end{thm}


\begin{exo}
  Montrer que $\Vert \cdot \Vert _{X'} $ est une norme.
\end{exo}

Si $X$ est un espace vectoriel normé complet (de Banach), alors $X'$ l'est aussi.

\begin{corollary}[Prolongement isométrique de $l$ sur $Y$]
  Soit $X$ espace vectoriel normé, $Y \subset X$ sous espace vectoriel de $X$ et $l \in Y'$ avec

  \[
  \Vert l \Vert = \sup_{ \substack{\Vert y \Vert \leq 1 \\ y \in Y} } \lvert \langle l,y \rangle  \rvert.
  \]

  Alors il existe un prolongement $L$ de $l$ de même norme

  \[
  \sup_{ \substack{x \in X \\ \Vert x \Vert \leq 1 } }  \lvert \langle l,x \rangle  \rvert = \sup_{ \substack{y \in Y \\ \Vert y \Vert \leq 1 } } \lvert \langle l,y \rangle  \rvert.
  \]
\end{corollary}

\begin{proof}
  Par le théorème de Hahn-Banach, on pose $p$ telle que $p(x) = \Vert l \Vert _{Y'} \Vert x \Vert  $ (l'application définie ainsi vérifie les propriétés de $p$ nécessaires à l'application du théorème).

  Par Hahn-Banach, il existe $L$ une forme linéaire sur $X$ telle que
  \[
  L(x)  = \langle L,x \rangle \leq p(x) = \Vert l \Vert _{Y'} \Vert x \Vert.
  \]

  Mais
  \[
  \langle L, -x \rangle \leq \Vert l \Vert _{Y'} \Vert -x \Vert,
  \]

  donc
  \[
  \lvert \langle L,x \rangle  \rvert \leq \Vert l \Vert _{Y'} \Vert x \Vert.
  \]

  Ainsi, en divisant par $\Vert x \Vert $, on obtient le résultat suivant :

  \[
  \forall x \in X, \left\lvert \left\langle L, \frac{x}{\Vert x \Vert } \right\rangle  \right\rvert \leq \Vert l \Vert _{Y'}.
  \]

  Or si on prend $x \in Y$,

  \[
  \left\lvert \left\langle L, \frac{x}{\Vert x \Vert } \right\rangle  \right\rvert \leq \Vert l \Vert _{Y'} = \sup_{ y \in Y \setminus \{ 0 \} } \left\lvert \left\langle L, \frac{y}{\Vert y \Vert } \right\rangle  \right\rvert.
  \]

  Comme $Y \subset X$ (ce qui entraîne que $\sup_{ y \in Y \setminus \{ 0 \} } \left\lvert \left\langle L, \frac{y}{\Vert y \Vert } \right\rangle   \right\rvert \leq \sup_{ x \in X \setminus \{ 0 \} } \left\lvert \left\langle L, \frac{x}{\Vert x \Vert } \right\rangle   \right\rvert  $), on a donc égalité, d'où l'isométrie.
\end{proof}

\begin{corollary}
  $\forall x_0 \in X$ espace vectoriel réel, il existe $L_0 \in X', \Vert L_0 \Vert _{X'} = \Vert x_0 \Vert _{X} $.
\end{corollary}

\begin{proof}
  $Y = \mathbb{R} x_0$. Soit $l(tx_0) \stackrel{\text{déf}}{=} t \Vert x_0 \Vert ^2 $ forme linéaire continue sur $Y$.

  Alors, en posant $t=1$, on obtient

  \[
  \Vert l \Vert _{Y'} =  \Vert x_0 \Vert
  \]

  %\left( \sup_{ y \in Y \setminus \{ 0 \} } \frac{\lvert \langle L,y \rangle  \rvert}{ \Vert y \Vert } = \frac{\Vert x_0 \Vert ^2 }{\Vert x_0 \Vert } \right) =

  et, par le théorème de Hahn-Banach,

  \[
  \Vert L_0 \Vert _{X'} = \Vert x_0 \Vert _{X}.
  \]
\end{proof}

\begin{exo}
  Traduire Hahn-Banach dans le cas où $X$ est un espace localement convexe.
\end{exo}

\begin{thm}[De Hahn-Banach, forme géométrique]\label{hb-geo}
  Soit $X$ espace vectoriel normé (ou espace localement convexe). Soient $A, B \subset X$ convexes et disjoints.

  \begin{enumerate}
    \item On suppose que $A$ est ouvert. Alors il existe un hyperplan affine (d'équation $\langle L,x \rangle = \text{constante}$) $\mathscr{H} $ qui sépare au sens large $A$ et $B$.
    \item Si $A$ est fermé, $B$ est compact, alors il existe $\mathscr{H} $ hyperplan qui sépare $A$ et $B$ au sens strict.
  \end{enumerate}
\end{thm}

\begin{figure}[h!]
  \centering
  \includegraphics[scale=0.3]{figures/hb-geo.png}
  \caption{$A = \{ x_1 \less 0 \}, B = \{ x_2 \geq 0 \}, \mathscr{H} = \{ x_1 = 0 \}  $.}
  \label{}
\end{figure}

\section{Espaces de Fréchet, topologies faible et faible \(\ast\)} \marginnote{26-09-2023}

\subsection{Topologies définies par une distance}

On rappelle la définition \ref{distance}.

\begin{definition}[Distances équivalentes]
  On dit que $d_1$ est équivalente à $d_2$ si et seulement si il existe \(C \bg 0\) tel que

  \[\frac{1}{C}d_1(x,y) \leq d_2(x,y) \leq C d_1(x,y).\]
\end{definition}

\(d_1 \sim d_2 \implies (X,d_1) \simeq (X,d_2)\), mais la réciproque est fausse.

\begin{exemple}
  On prend un espace métrique \((X,d)\) avec les distances

  \[\delta(x,y) = \frac{d(x,y)}{1+ d(x,y)} \text{ et } \delta'(x,y) = \inf(1, d(x,y)).\]

  Ces distances sont équivalentes entre elles.%, mais elles ne sont pas équivalentes à \(d\).
\end{exemple}

\begin{proof}

  \

  \begin{enumerate}
    \item Montrons que \((X, d) \sim (X, \delta')\). On remarque d'abord que

    \[\delta(x,y) = \frac{d(x,y)}{1+ d(x,y)} \leq d(x,y), \]

    ce qui veut dire que \((X, d) \prec (X, \delta)\) (car si \(\mathcal{O}\) est un ouvert pour \(\delta\), alors il le sera forcément pour \(d\)).

    Prenons
    \begin{equation}\label{bij-dist}
      f(t) = \frac{t}{1+t}.
    \end{equation}

    La fonction \(f\) est une bijection de \(\mathbb{R} ^{+}\) dans \([0, 1]\). En effet, montrons qu'il existe \(g : [0, 1] \to \mathbb{R} ^{+}\) telle que $g \circ f = f \circ g = \operatorname{id}$.

    On a

    \begin{gather*}
      \frac{t}{1+t} = s \implies t = ts+s \implies t = \frac{s}{1-s}.
    \end{gather*}

    Donc \(d(x,y) = \frac{\delta(x,y)}{1 - \delta(x,y)}\). Donc si \(d(x,y) \less \varepsilon\), alors \(d(x,y) \less \frac{\varepsilon}{1- \varepsilon}\). Donc \((X,\delta) \prec (X,d)\).

    \item Montrons que \(\delta \sim \delta'\).

    On a \[\delta = \frac{d}{1+d} \leq \begin{cases}
      1 \\
      \delta.
    \end{cases}\]

    En effet, cela vient du fait que \(\frac{d(x,y)}{1+d(x,y)} \underset{d(x,y) \to \infty }{\longrightarrow} 1\). Donc \[\delta(x,y) \leq \delta'(x,y).\]

    Mais \(\delta' \leq 2 \delta\). En effet, on distingue deux cas :

    \begin{enumerate}
      \item Si \(\delta \leq  1 \text{ et } \delta' = d, \text{ alors } d \leq 2 d \),
      \item Si \(\delta \geq  1 \text{ et } \delta' = 1, \text{ alors } 1 \leq 2 d \).
    \end{enumerate}

    \item Montrons que $\delta$ est une distance.

    \begin{enumerate}
      \item Montrons l'inégalité triangulaire. Si $d(x,y) \leq d(x,z) + d(z,y)$, montrons que $\delta(x,y) \leq \delta(x,z) + \delta(z,y)$.

      Est-ce que \(f(d(x,y)) \leq f(d(x,z))+ f(d(z,y))\), avec \(f\) définie dans \ref{bij-dist} ?

      \begin{enumerate}
        \item $f$ est croissante, donc \(f(d(x,y)) \leq f[d(x,z) + d(z,y)]\). Il suffit de voir que $f(t) \leq f(u)+f(v)$.

        \item Montrons la sous-additivité de $f$. Posons \[v  \mapsto \varphi(v) =  f(u+v) - f(u) - f(v). \]

        On a \(\varphi(0) = 0\), car \(f(0) =0\) et \(\varphi(v) = f'(u+v) - f'(v) \less 0\), car $f$ est une fonction croissante.
      \end{enumerate}
    \end{enumerate}
  \end{enumerate}
\end{proof}

\emph{Sous quelles conditions un espace localement convexe est métrisable ?}

On remarque par exemple que $\mathscr{F}([0, 1], \mathbb{R})$ muni de la topologie de la convergence simple n'est pas métrisable. Plus généralement, les topologies faibles ne sont pas métrisables, sauf si on travaille en dimension finie. Par ailleurs, \(X\) muni de la topologie grossière n'est pas métrisable (non séparée).

\begin{prop}\label{elc-metr}
  Soit \(X\) un espace localement convexe (donc séparé). Alors les assertions suivantes sont équivalentes.

  \begin{enumerate}
    \item \(X\) est métrisable.
    \item Il existe une base dénombrable de voisinages de 0 dans \(X\), et ce pour tout \(x \in X\).
    \item La topologie de \(X\) est engendrée par une famille dénombrable de semi-normes.
  \end{enumerate}
\end{prop}

\begin{proof}
  \begin{enumerate}
    \item \((1) \implies (2)\). La topologie sur \(X\) est équivalente à \((X,d)\). Soit \((X,d)\) un espace métrique. Il suffit de poser

    \[\mathcal{O} _{\frac{1}{n}} = \left\{ x \mid d(x,0) \less \frac{1}{n}\right\} \ (\mathbb{R} \text{ est archimédien}). \]

    Alors \(\forall \varepsilon \bg 0, \exists n \text{ tel que } \mathcal{O} _{\frac{1}{n}} \subset \mathcal{O} _{\varepsilon}\). Donc \(x + \mathcal{O} _{\frac{1}{n}}\) est une base dénombrable de voisinages de \(x\).

    \item \((2) \implies (3)\). On sait que \(\mathscr{T}\) topologie de \(X\) est donnée par une famille de semi-normes. Les voisinages de 0 dans \(X\) sont donnés par

    \[\mathcal{O} _{a, \varepsilon} = \bigcap _{i=1} ^{n} \mathcal{O} _{\varepsilon, a_i}, \text{ avec } i \in \{ 1, \dots, n \}.\]

    On rappelle que \(\mathcal{O} _{\varepsilon,a_i} = \{ x \mid \rho _{a_i} \less \varepsilon \}\).

    On peut choisir \(\varepsilon = \frac{1}{n}\). On sait qu'il existe une base dénombrable de voisinages de 0 dans \(X\). Soit \(U_n\) une base de voisinages dénombrable de 0. On pose

    \[\rho_n(x) = \mu _{U_n}(x).\]

    On prend les \(U_n\) convexes, balancés, absorbants comme dans le théorème \ref{convexes-balances} (c'est possible, car \(X\) est un espace localement convexe).

    \item \((3) \implies (1)\).

    \begin{enumerate}
      \item Soit \( (\rho_n)\) une famille dénombrable de semi-normes sur \(X\). On pose

      \[d(x,y) = \sum_{n=1}^{\infty} 2 ^{-n} \frac{\rho_n(x-y)}{1 + \rho_n(x-y)}.\]

      Montrons que \((X, \text{ELC}) \prec (X, d)\). Soit \(U \in \mathscr{T}\) (la topologie ELC). On se ramène aux voisinages de 0. On a

      \[U = \bigcap _{\text{finie}} \mathcal{O} _{\varepsilon, a}, a \in A. \]

      Comme il existe une base dénombrable de voisinages, on peut choisir

      \[U _{\varepsilon} = \bigcap _{j=1} ^{N} = \{ x \mid \rho _{j}(x-0)\less \varepsilon \}, \text{ avec } A= \mathbb{N}.\]

      Ce voisinage est inclus dans \(\left\{ x \mid \sum_{}^{} \rho_j(x-0) \leq N \varepsilon \right\}\).

      Montrons que \(U\) est un voisinage de \(x\) pour la topologie métrique \((X,d)\).

      Soit \(\varepsilon \bg 0\). On pose \(d(x,0) = \left( \sum_{1}^{N} + \sum_{N+1}^{\infty}\right) \frac{\rho_n}{1+ \rho_n}\). Or \(N\) est tel que

      \[\sum_{N+1}^{\infty} 2 ^{-n} \less \varepsilon \implies \sum_{N+1}^{\infty} 2 ^{-n} \frac{\rho_n}{1+ \rho_n} \less \varepsilon.\]

      De plus,

      \begin{gather}
        d(x,y) \leq \varepsilon + \sum_{n=1}^{N}  \frac{d_n}{1+d_n} \less \varepsilon + \sum_{n=1}^{N} d_n(x,y). \label{vois_un}
      \end{gather}

      Or \(\rho_n(x-y)\less \varepsilon\), car \(x \in y + U _{\varepsilon}\).

      Donc \ref{vois_un} devient

      \begin{gather*}
        d(x,y) \leq \varepsilon + N \varepsilon \text{ avec } N \text{ fixé.}
      \end{gather*}

      Donc \(\mathscr{T} \prec (X,d)\).

      \item Montrons que \((X,d) \prec \mathscr{T}\). On doit majorer \(\rho_m(x-y)\).

      Or \begin{gather*}
        d(x,y) = \sum_{n=1}^{\infty} 2 ^{-n} \frac{\rho_n(x-y)}{1+ \rho_n(x-y)} \geq 2 ^{-m} \frac{\rho_m(x-y)}{1 + \rho_m(x-y)}.
     \end{gather*}

     Et

     \begin{gather*}
       2 ^{m}d(x,y) \geq \frac{\rho_m(x-y)}{1 + \rho_m(x-y)} \geq  f(t).
     \end{gather*}

     Donc on a \(\rho_m(x,y) \leq  g(2 ^{m} d(x,y))\), où \(g\) est la réciproque de \(t \mapsto \frac{1}{1+t}\).
    \end{enumerate}
  \end{enumerate}
\end{proof}

\begin{prop}
  Soit \(X\) un espace localement convexe qui vérifie l'une des propriétés énoncées dans la proposition \ref{elc-metr} (i. e. métrisable). On note la topologie de \(X\) ELC par \(\mathscr{T}\). Alors \(X\) est complet pour \(\mathscr{T}\) si et seulement si \((X,d)\) est complet.
\end{prop}

\begin{proof}
  Cette proposition se démontre exactement comme \ref{elc-metr}.
\end{proof}

\begin{definition}
  Soit \(X\) un espace localement convexe. On dit que \(X\) est un \textbf{espace de Fréchet} si \(X\) est \textbf{métrisable et complet}.
\end{definition}

\begin{exemple}

  \

  \begin{enumerate}
    \item \emph{Les espaces localement convexes qui ne sont pas des Fréchet. }
    \begin{enumerate}
      \item \emph{Non métrisables.} \(\mathscr{F}([0, 1], \mathbb{R}) \) muni de la topologie de la convergence simple, les topologies faibles, \(\dots\)
    \end{enumerate}

    \item \emph{Les espaces localement convexes qui sont des Fréchet. } Les espaces de Banach, par exemple \(\mathscr{F}([0, 1], \mathbb{R}) \) muni de la topologie de la convergence uniforme, \(\mathcal{C}^{\infty}_0(K)\), \(\dots\)
  \end{enumerate}
\end{exemple}

\subsection{Espace de Schwarz \(\mathscr{S}(\mathbb{R} ^{d}) \subset \mathcal{C}^\infty(\mathbb{R}^d)\)}

\(\varphi \in \mathscr{S}(\mathbb{R}^d) \iff \rho _{\alpha, \beta}(\varphi) = \sup_{ \mathbb{R} } \left\lvert x ^{\alpha} D _{\varphi} ^{\beta} \right\rvert \less \infty\).

Montrons que \(\mathscr{S}(\mathbb{R})\) est complet. On va regarder \(\rho _{0, 0}, \rho _{0, 1}, \rho _{1, 0}, \rho _{1, 1}, \dots\)

\begin{enumerate}
  \item \(\rho _{0, 0}(\varphi _{p+q} - \varphi _{p}) \less \varepsilon\), donc \( \varphi _{p} \longrightarrow \varphi\), donc

  \[\sup_{ \mathbb{R} } \lvert \varphi _{p+q} (x) - \varphi _{p}(x) \rvert \less \varepsilon. \]

  En particulier pour tout \(K \subset \mathbb{R}\), \(\varphi_n\) est de Cauchy dans \(\mathcal{C}^0(K)\). Or \(\mathcal{C}^0(K)\) est complet, donc \(\varphi_n \underset{\text{uniformément}}{\longrightarrow} \varphi.\) Comme \(K\) est arbitraire, elle converge localement sur tout \(\mathbb{R}\). On a

  \[\rho _{0, 0}(\varphi _{p+q} - \varphi_p) \less \varepsilon, \]

  donc \[\rho _{0, 0}(\varphi - \varphi_p) \less \varepsilon.\]

  Donc \(\varphi _{p}\) converge pour \(\rho _{0,0}\).

  \item On a besoin de rappeler le lemme suivant :

  \begin{lemma}
    Si \(\varphi' \longrightarrow \psi\) uniformément et \(\varphi_n \longrightarrow \varphi\) simplement, alors \(\psi = \varphi'\).
  \end{lemma}
\end{enumerate}



\subsection{Topologie forte et topologie faible sur les espaces de Banach}\marginnote{10-10-2023}

Soient \((E, \left\Vert \cdot \right\Vert) \) un espace de Banach (réel ou complexe) et \((E', \left\Vert \cdot \right\Vert')\) son dual topologique. On rappelle que

\[E' = \{ l \in \mathscr{L}(E, \mathbb{R}), \exists C \bg 0, \forall x \in E, \left\lvert \langle l,x \rangle  \right\rvert \leq C \left\Vert x \right\Vert^2 \}, \]

avec la norme sur le dual définie dans \ref{norm-dual}.

\((E', \left\Vert \cdot \right\Vert')\) est un espace de Banach, un cas particulier de \(\mathscr{L}(E,F)\), avec pour \(u \in \mathscr{L}(E,F)\),

\[\left\Vert u \right\Vert _{\mathscr{L}(E,F)} = \sup_{\substack {x \in E \\\left\Vert x \right\Vert_E \leq 1  }}  \left\Vert u(x) \right\Vert_F. \]

On affaiblit \((E,\left\Vert \cdot \right\Vert )\). Alors \(\sigma(E,E')\) est la topologie la moins fine qui rend continue toutes les formes linéaires sur \(E\). \(X \sim \sigma(E,E')\) est muni des semi-normes \(\left\lvert \langle l,x \rangle  \right\rvert = \rho_l(x)\). Un voisinage de 0 est défini de la manière suivante :

\[\mathcal{O} _{\underline{l}, \varepsilon} = \{ x \in E : \sup_{ 1 \leq  i \leq n} \langle l_i, x \rangle \less \varepsilon \}, \underline{l} = (l_1, \dots, l_n).\]

\subsection{Comparaison des topologies \((E, \left\Vert \cdot \right\Vert)\) et \(\sigma(E,E')\)}

\begin{lemma}
  Soit \(E\) un espace de Banach. Alors la norme définie

  \[x = \sup_{\substack {l \in E'\\ \left\Vert l \right\Vert' \leq 1 }} \left\lvert \langle l,x \rangle  \right\rvert = \langle l_0,x \rangle.  \]

  est telle que le \(\sup\) est atteint. On a \(\sup = \max\).
\end{lemma}

\begin{proof}
  On a \(x \neq 0\) par la définition de \(\left\Vert \cdot \right\Vert'\). Alors

  \[\left\lvert \left\langle l,\frac{x}{\left\Vert x \right\Vert } \right\rangle  \right\rvert \leq \left\Vert l \right\Vert' \leq 1.\]

  Alors \[\left\lvert \langle l,x \rangle  \right\rvert \leq  \left\Vert x \right\Vert \text{ pour tout } l \in E' \text{ tel que } \left\Vert l \right\Vert' \leq 1.\]

  Soit \(x_0\) et \(F\) tel que \(F = \mathbb{R} x_0\). Alors \(\forall \lambda \in \mathbb{R}, l_0(\lambda x_0) = \lambda\) et \(\left\Vert l_0 \right\Vert = \left\Vert x_0 \right\Vert.\) Par Hahn-Banach, on peut prolonger \(l_0\) en \(L_0\) sur tout l'espace de Banach.
\end{proof}

\begin{prop}
  \[(E,\left\Vert \cdot \right\Vert) \prec \sigma(E,E').\]

  Donc \(X \sim \sigma(E, E')\) est un espace localement convexe.
\end{prop}

\begin{proof}
  On a

  \[\rho_l(x) = \left\lvert \langle l,x \rangle  \right\rvert \leq \left\Vert l \right\Vert' \left\Vert x \right\Vert.  \]
\end{proof}

\begin{proof}[Autre démonstration]
  Montrons que \(\mathcal{O} _{\underline{l}, \varepsilon}\) est un ouvert de \((E, \left\Vert \cdot \right\Vert)\), i. e. \(\left\Vert x \right\Vert \less \delta\). On prend \(n\) formes linéaires \(l_i, i \in \{ 1, \dots, n \}\) et on considère

  \[ \left\Vert x \right\Vert = \sup_{ \left\Vert l \right\Vert' \leq 1  } \left\lvert \langle l,x \rangle  \right\rvert. \]

  Or \(\left\lvert \langle l_i,x \rangle  \right\rvert \less \varepsilon\) ... (à suivre).
\end{proof}

\begin{proof}
  Montrons que \(\sigma(E,E')\) est séparé. Soient \(x_1, x_2\) distincts. Montrons qu'il existe \(\mathcal{O}_1\) et \(\mathcal{O}_2\) de \(\sigma(E,E')\) tels que \(\mathcal{O}_1 \cap \mathcal{O}_2 = \emptyset\).

  Par le théorème de Hahn-Banach \ref{hb-geo}, pour \(A = \{ x_1 \}, B = \{ x_2 \} \) compacts et convexes, pour tout \((x,y) \in A \times B\), on a

  \[\langle l,x \rangle \less \alpha \less \langle l,y \rangle.\]

  Donc

  \[\langle l,x_1 \rangle \less \alpha \less \langle l,x_2 \rangle.\]

  On a \(x_1 \in \mathcal{O} _{\alpha,l}^{1} =\{ x : \langle l,x \rangle \less \alpha\}\) et \(\mathcal{O} _{\alpha,l} ^{2} = \{ y : \langle l,y \rangle \bg \alpha\}\), ces ouverts séparent \(x_1\) et \(x_2\).
\end{proof}


\begin{thm}
  \((E, \left\Vert \cdot \right\Vert )\) est strictement plus fine que \(\sigma(E, E')\) sauf en dimension finie.
\end{thm}

\begin{proof}
  On considère \(S = \{ \left\Vert x \right\Vert =1 \}\). Alors \(S = \overline{S}\), son adhérence.

  Soit \(x_0\) de norme plus petite que 1. Montrons que pour tout \(V\) voisinage de 0 dans \(\mathscr{T}\), \(V \cap S \neq \emptyset\). On a

  \[V = \{ x : \left\lvert \langle l_i,x - x_0 \rangle  \right\rvert \less \varepsilon, 1 \leq i \leq n \}.\]

  Comme \(\operatorname{dim}(E) = \infty\), il existe \(y_0 \neq 0\) tel que \(\langle l_i, y_0 \rangle =0, \forall i\). On a

  \begin{gather*}
    g(t) = \left\Vert x_0 + t y_0 \right\Vert.
  \end{gather*}

  On a \(g(0) = \left\Vert x_0 \right\Vert \less 1\) et \(g(\infty) = +\infty\). La fonction \(g\) est continue, donc il existe \(t_0 \in (0, \infty)\) tel que \(\left\Vert x_0 + t_0 y_0 \right\Vert = 1\). Donc \(x_0 + t_0 y_0 \in S\) et \(x_0 + t_0 y_0 \in V\), car

  \[\left\lvert \langle l_i, (x_0 + t_0 y_0) - x_0 \rangle  \right\rvert = \left\lvert \langle l_i, t_0 y_0 \rangle  \right\rvert = \left\lvert t_0 \langle l_i, y_0 \rangle  \right\rvert = 0 \less \varepsilon.\]
\end{proof}

\begin{remark}
  \(\forall t \in \mathbb{R}, \langle l_i, x_0 + t y_0 \rangle = 0\). Alors \(V\) contient toute une droite.
\end{remark}

\begin{remark}
  Pour \(E\) Banach séparable, \(B_E\) boule unité fermée de \((E,\left\Vert \cdot \right\Vert)\) est métrisable pour \(\sigma(E,E')\).

  Si \(E\) est réfléxif (c'est-à-dire que l'injection naturelle dans son bidual est surjective), alors \(B_E = \{ \left\Vert x \right\Vert \leq 1 \} \) est un espace métrique compact pour \(\mathcal{O}(E, E')\).
\end{remark}

\begin{exemple}[Wikipédia]
  On considère la convergence forte et la convergence faible dans l'espace \(L ^2\) (qui est un espace de Hilbert d'après \ref{l2-hilbert}). La convergence forte de \(\psi_n\) vers un élément \(\psi \in L ^2(\mathbb{R}^n)\) signifie :

  \[\int_{\mathbb{R}^n} \left\lvert \psi_n - \psi \right\rvert ^2 d \mu \underset{n \to \infty}{\longrightarrow}  0.\]

  La notion de convergence forte dans \(L ^2\) correspond à celle de la norme dans \(L ^2\). En revanche, pour que la suite \(\psi_n\) converge faiblement, il suffit que

  \[\int_{\mathbb{R}^n} \overline{\psi_n}f d \mu \longrightarrow \int_{\mathbb{R}^n} \overline{\psi} f d \mu\]

  pour toute fonction \(f \in L ^2\). Par exemple, dans \(L ^2((0, 2 \pi))\), la suite de fonctions

  \[\psi_n(x) = \sqrt{\frac{2}{\pi}}\sin(nx)\]

  forme une base orthonormée. La limite forte de \(\psi_n\) n'existe pas. Mais par le lemme de Riemann-Lebesgue, la limite faible existe et vaut 0.
\end{exemple}

\subsection{Théorème de Banach-Steinhaus, suites faiblement et fortement convergentes}

\begin{thm}
  Soient \(E,F\) espaces de Banach et \((T_a) _{a \in A} \in \mathscr{L}(E,F)\) telle que

  \[\forall x \in E, \sup_{ a \in A } \left\Vert T_a x \right\Vert _{F} \less + \infty.\]

  Alors \[\sup_{ a \in A } \left\Vert T_a \right\Vert \less +\infty \text{ (bornée en norme),}\]

  i. e. \(\exists C \bg 0, \forall x \in E, \forall \alpha \in A, \left\Vert T_\alpha(x)\right\Vert \leq C \left\Vert x \right\Vert\).
\end{thm}


\begin{corollary}
  Soit \(T_n \in \mathscr{L}(E,F)\) avec \(\left\Vert T_n \right\Vert \less +\infty\) avec \(\forall x \in E, T_n x \underset{n \to \infty}{\longrightarrow}y \in F\). On note \(y = Tx\). Alors \(T \in \mathscr{L}(E,F)\) et \(T = \lim \inf  T_n\).
\end{corollary}

\begin{proof}
  Montrons que \(T_n\) est linéaire. En effet,

  \begin{gather*}
    T_n(\lambda x + \lambda' x') = \lambda T_n(x)+ \lambda'T_n(x').
  \end{gather*}

  Par passage à la limite, on obtient \(T(\lambda x+ \lambda' x') = \lambda T(x) + \lambda' T(x')\).

  Montrons l'autre partie du corollaire. Par Banach-Steinhaus, si on considère \(A = \mathbb{N}\), pour tout \(x \in E\), \(T_n x\) est convergente, donc bornée, i. e. \(\left\Vert T_n x \right\Vert \less \infty\), donc \(\displaystyle\sup_{ n } T_n \less C\) comme \(\left\Vert T_n x \right\Vert \leq \left\Vert T_n \right\Vert \left\Vert x \right\Vert\).

  Par passage à la limite, on obtient \(\left\Vert T x \right\Vert \leq C \left\Vert x \right\Vert\), avec \(C = \lim \inf \left\Vert T_n \right\Vert\).
\end{proof}

\begin{definition}
  Soit \(E\) espace de Banach. \(B \subset E\) est bornée si et seulement si \[\exists C \geq 0, \forall x \in B, \left\Vert x \right\Vert \leq C. \]
\end{definition}

\begin{corollary}
  \(B \subset E\) est bornée si et seulement si \(\forall l \in E', l(B) \subset \mathbb{R}\) est borné.
\end{corollary}

\subsection{Suites faiblement et fortement convergentes}

\begin{definition}[Suite fortement convergente]
  Soit \(x_n \text{ une suite de }  E\). On dit que \(x_n\) est fortement convergente lorsque \(x_n \longrightarrow x \iff \left\Vert x_n -x \right\Vert \longrightarrow 0\).
\end{definition}

\begin{definition}[Suite faiblement convergente]
  Soit \(x_n \text{ une suite de }  E \). On dit que \(x_n\) est faiblement convergente lorsque \(x_n \longrightarrow x \iff \forall l \in E', \langle l,x_n \rangle \longrightarrow \langle l,x \rangle\).

  On note alors \(x_n \stackrel{\text{w}}{\longrightarrow} x\) (avec \emph{weak} qui signifie faible en anglais) ou \(x_n \rightharpoonup x\).
\end{definition}


\begin{corollary}

  \

  \begin{enumerate}
    \item Si \(x_n \longrightarrow x\) dans \((E, \left\Vert \cdot \right\Vert)\), alors \(x_n \longrightarrow x\) dans \(\sigma(E,E')\).
    \item Si \(x_n \longrightarrow x\), alors \(x_n\) est bornée dans \(E\) et \(\left\Vert x \right\Vert_E \leq \lim \inf \left\Vert x_n \right\Vert\).
    \item Si \(x_n \longrightarrow x\) et \(l_n \longrightarrow l\), alors \(\langle l_n,x_n \rangle \longrightarrow \langle l,x \rangle\).
  \end{enumerate}
\end{corollary}

\begin{proof}

  \

  \begin{enumerate}
    \item Soit \(l \in E'\). Alors

    \[\left\lvert \langle l,x_n - x \rangle  \right\rvert \leq \left\Vert l \right\Vert' \left\Vert x_n -x \right\Vert  \underset{n \to \infty}{\longrightarrow} 0.\]

    \item \(T_n : E \longrightarrow \mathbb{R}\). Alors on définit

    \[T_n l = \langle l, x_n \rangle \longrightarrow \langle l,x \rangle =  T l,  \]

    car \(x_n\) tend faiblement vers \(x\). Alors d'après le corollaire, \(\sup \left\Vert T_n \right\Vert \less \infty\), avec \(T \in (E')' = E''\) et \(\left\Vert T \right\Vert'' \leq \lim \inf \left\Vert x_n \right\Vert\), donc \(\left\lvert T l \right\rvert = \left\lvert \langle l,x \rangle  \right\rvert \) par passage à la limite.

    \item \[\langle l_n,x_n \rangle - \langle l,x \rangle = \langle l, x_n - x \rangle + \langle l, x_n - x \rangle.\]

    Or \(\langle l, x_n - x \rangle \longrightarrow 0\), car \(x_n \longrightarrow x\) et \(\langle l, x_n - x \rangle \longrightarrow 0\), car \(\left\lvert \langle l_n -l, x_n \rangle  \right\rvert \leq  \left\Vert l_n - l \right\Vert \left\Vert x_n \right\Vert \longrightarrow 0 \).
  \end{enumerate}
\end{proof}



\subsection{Topologie faible \(\ast\)}\marginnote{13-10-2023}

On considère \(E\) espace de Banach avec \((E, \left\Vert \cdot \right\Vert ) \prec \sigma(E,E')\). On construit la topologie \(\ast \sigma(E', E)\) une topologie sur \(E'\).

La topologie forte sur \(E' = F\) est donnée par la norme

\[\left\Vert l \right\Vert' = \sup_{\left\Vert x \right\Vert \leq 1} \left\lvert \langle l,x \rangle  \right\rvert.\]

Sur \(E\), on a aussi \(\sigma(F, F') = \sigma(E',E'')\). Si \(E = E''\) (i. e. \(E\) est réfléxif), alors la topologie \(\sigma(E',E'')\) se confond avec la topologie faible \(\sigma(E,E')\). Par contre, si \(E\) s'injecte dans \(E''\), on a besoin de définir une autre topologie \(\ast \sigma(E',E)\) moins fine que \(\sigma(E',E'')\).

\begin{prop}
  Il existe une isométrie \(J : E \hookrightarrow E''\).
\end{prop}

\begin{proof}
  On définit une application \(J : E \longrightarrow E''\) telle que

  \[\langle Jx,x' \rangle \stackrel{\text{déf}}{=} \langle x',x \rangle.\]

  Montrons que \(\left\Vert Jx \right\Vert' = \left\Vert x \right\Vert\).

  On a besoin d'introduire le résultat suivant :

  \begin{lemma}
    \[\left\Vert x \right\Vert = \sup_{\left\Vert l \right\Vert' \leq 1} \left\lvert \langle l,x \rangle  \right\rvert = \max _{\left\Vert l \right\Vert' \leq 1} \left\lvert \langle l,x \rangle  \right\rvert.\]
  \end{lemma}

  Donc \(\left\Vert Jx \right\Vert = \left\Vert x \right\Vert\) en prenant

  \[\sup_{\substack{x' \in E'\\ \left\Vert x' \right\Vert \leq 1}} \left\lvert \langle x',x \rangle  \right\rvert = \left\Vert x \right\Vert.\]
\end{proof}

\begin{remark}
  \(J\) n'est pas unitaire (non surjectif si \(E\) n'est pas réfléxif).
\end{remark}

\begin{exemple}
  On considère \(E = L ^{1}\). Soit \(f \in L ^{1}\), alors \(J f \in L ^{\infty}\) et on a :

  \[\langle Jf, g \rangle = \langle g,f \rangle  = \int g(x)f(x)dx \ (f : \mathbb{R}^d \longrightarrow \mathbb{R}).\]

  Mais \(g(0)\) ne peut pas s'écrire comme une intégrale \(\int_{}^{} g(x)f(x) dx, \forall f \in L ^{1}\) si \(g \in C _{0}^{0} (\mathbb{R}^d) \subset L ^{\infty}\).
\end{exemple}

\begin{prop}
  \(\ast \sigma(E',E)\) est séparée.
\end{prop}

\begin{proof}
  Soient \(l_1, l_2 \in E'\) tels que \(l_1 \neq l_2\), alors il existe \(x \in E\) tel que \(l_1(x) \neq l_2(x)\), donc

  \[\langle l_1,x \rangle \neq \langle l_2,x \rangle,\]

  ce qui implique que \(\langle l_1,x \rangle \less \alpha \less \langle l_2,x \rangle\).

  On a \(\mathcal{O}_1 = \{ l \in E' : \langle l,x \rangle \less \alpha\}\) ouvert de \(E'\) et \(\mathcal{O}_2 = \{ l \in E', \langle l,x \rangle \bg \alpha\}\).
\end{proof}

\begin{remark}
  Pour \(x\) fixé, l'application \(l \longmapsto \left\lvert \langle l,x \rangle  \right\rvert\) semi-norme de \(E'\).
\end{remark}

\begin{prop}[Autres propriétés]
  \(\ast \sigma(E',E)\) n'est pas métrisable.
\end{prop}

\begin{remark}
  \(E\) est séparable et réfléxif si et seulement si \(E'\) est séparable et réfléxif.
\end{remark}

\begin{thm}
  Si \(E\) est un Banach séparable, alors la boule unité fermée

  \[B_E = \{ \left\Vert l \right\Vert \leq 1 \}\]

  est métrisable pour \(\ast \sigma\).
\end{thm}

\begin{prop}
  Soit \(\xi : E' \longrightarrow \mathbb{R}\) avec (\(\xi \in E''\)). Si \(\xi\) est continue pour \(\ast \sigma(E,E')\), il existe \(x \in E, \xi = Jx\) et

  \[\langle \xi, l \rangle = \langle l,x \rangle.\]
\end{prop}

\begin{thm}[De représentation de Riecz]
  Si \(\xi \in \mathscr{H}' = \mathscr{H}\), alors

  \[\forall l \in \mathscr{H}, \langle \xi, l \rangle = \langle l,x \rangle.\]
\end{thm}



\chapter{Théorie de distributions}\marginnote{17-10-2023}

La théorie de distributions utilise une grande variété des fonctions test.

Ainsi une mesure de Radon \(\mu\) sur un espace localement compact \(\Omega\) (par exemple un ouvert de \(\mathbb{R}^{d}\)) est une distribution d'ordre 0 agissant \(\mathcal{C}^{0}_{0}\) (noté encore \(\mathcal{K}(\Omega)\)) notamment par

\[\langle \mu, f \rangle = \int_{\Omega} f(x)d \mu(x).\]

%Soit \(a _{\varphi} \in \mathcal{C}^{\infty}_{0}(\Omega)\), un espace de fonctions \(\mathcal{C}^{\infty}\) à support compact.

L'exemple le plus couramment utilisé d'une distribution d'ordre 0 est la mesure de Dirac \(\delta _{x_0}\).

Sur \(\Omega = \mathbb{R}^{d}\), il suffit de prendre l'espace de Schwarz \(\mathcal{S}(\mathbb{R}^{d})\) qui est un espace de Fréchet. Par contre si \(\Omega\) est un ouvert borné, il y a des problèmes sur les bords de \(\Omega\).

Les distributions d'ordre 1 agissent quant à elles par \(\langle \delta _{x_0}',f \rangle = -f'(x_0)\) dans \(\mathcal{C}^{1}_{0}\) qui ne sont ni des espaces de Banach, ni des espaces de Fréchet. On choisit généralement \(\mathcal{C}^{0}_{0}(\Omega)\).

Les espaces fonctionnels sont rangés en deux catégories :

\begin{itemize}
  \item Les espaces de Lebesgue ;
  \item Les espaces de fonctions différentiables.
\end{itemize}
{\fontencoding{U}\fontfamily{futs}\selectfont\char 66\relax}
\section{Espaces de Lebesgue}

\subsection{Mesure de Lebesgue}

Il est important de rappeler la notion d'espace mesuré.

\begin{definition}[Rappel : tribu]
 Soit \(X\) un ensemble. On dit qu'une collection d'ensembles \(\mathcal{T}\) est une tribu si

 \begin{enumerate}
   \item \(X \in \mathcal{T}\) et \(\emptyset \in \mathcal{T}\) ;
   \item Si \(A \in \mathcal{T}\), alors \(A ^{C} \in \mathcal{T}\) ;
   \item Si \((A_n)_{n \in \mathbb{N}}\) est une suite de \(\mathcal{T}\), alors

   \[\bigcup _{n \in \mathbb{N}} A_n \in \mathcal{T}.\]
 \end{enumerate}

 On dit que \((X, \mathcal{T})\) est un espace mesurable.
\end{definition}

\begin{remark}
  \(\mathcal{T}\) est aussi stable par intersection dénombrable, l'intersection étant complémentaire à la réunion...
\end{remark}

\begin{definition}[Mesure]
  Soit \((X, \mathcal{T})\) un espace mesurable. On dit qu'une application \(\mu : X \longrightarrow [0, \infty)\) est une mesure si :
  \begin{enumerate}
    \item \(\mu(\emptyset) = 0\) ;
    \item Pour toute suite \((A_n)\) de \(\mathcal{T}\) disjointe, on a

    \[\mu\left(\bigcup _{n \in \mathbb{N}} A_n\right) = \sum_{n \in \mathbb{N}} \mu(A_n). \]
  \end{enumerate}

  On dit alors que \((X, \mathcal{T}, \mu)\) est un espace mesuré.
\end{definition}

\begin{exemple}
  Pour \(\mathbb{R}^{d}\), \(\mathcal{T}\) est la tribu borélienne engendrée par les pavés \(\displaystyle \prod_{i=1}^{d} [a_i,b_i)\), avec \(a_i, b_i \in \mathbb{R}\). On peut aussi l'engendrer par les ``quadrants'' \(\displaystyle \prod_{i=1}^{d} [a_i, \infty) \).

  La mesure de Lebesgue se calcule comme suit. Si \(d=1\), alors \(\mu([a,b)) = b-a\). Pour les pavés, on aura :

  \[\mu\left( \prod_{i=1}^{d} [a_i, b_i) \right) = \prod_{i=1}^{d}(b_i - a_i).\]

  Elle se caractérise par le fait d'être stable par translation.
\end{exemple}

Pour avoir une théorie cohérente, il faut compléter \(B(R ^{d}) \longrightarrow \overline{\mathcal{T}}\) (la tribu borélienne) en ajoutant des ensembles négligeables. Ainsi \(\overline{\mathcal{T}}\) est la plus petite tribu contenant \(\mathcal{T}\) et les ensembles négligeables.

\begin{definition}
  \(A \in \mathbb{R}^{d}\), i. e. \(A\) est mesurable si et seulement si \(A \in \overline{\mathcal{T}}\).
\end{definition}

\begin{remark}
  A toute fin utile, on considérera que tous les ensemble sont mesurables.
\end{remark}

\emph{Comment mesurer les fonctions \(\mu(F)\) ?}

On peut utiliser :

\begin{itemize}
  \item L'intégrale de Riemann ;
  \item L'intégrale de Lebesgue.
\end{itemize}

\subsection{Intégrale des fonctions positives}

Soit \((X, \mathcal{T}, \mu)\) un espace normé mesuré.

\begin{definition}[Fonction mesurable]
  On dit que \(f : X \longrightarrow \mathbb{R}\) est mesurable si pour tout borélien \(B\) de \(\mathbb{R}\), on a \(f ^{-1}(B) \in \mathcal{T}\).
\end{definition}

\begin{prop}[Axiomes]
  Il existe une application définie sur l'ensemble mesurable des fonctions mesurables positives de \(\mathbb{R}^{d}\), à valeurs dans \(\mathbb{R}\), notée \(f \longmapsto \int_{}^{} f(x)dx \) qui réalise les propriétés suivantes :

  \begin{enumerate}
    \item \emph{Linéarité :} pour tous \(\alpha, \beta \geq  0\), on a

    \[\int_{}^{}(\alpha f(x)+ \beta g(x)) dx = \alpha \int f(x) dx + \beta \int g(x)dx. \]

    \item \emph{Croissance :} si \(\forall x, f(x) \leq g(x)\), alors

    \[\int_{}^{} f(x) dx \leq \int_{}^{} g(x) dx.  \]

    \item \emph{Normalisation :} pour tout pavé \(A = \prod_{i=1}^{d} [a_i, b_i)\), on a

    \[\int_{}^{} \mathds{1}_{A}(x)dx = \mu(A).\]

    \item \emph{Théorème de Beppo-Levi (ou de convergence monotone) :} si \((f_n)\) est une suite croissante de fonctions mesurables, alors

    \begin{equation}\label{Beppo-Levi}
      \underbrace{\int_{}^{} \lim_{n \to \infty} f_n(x)dx}_{\int_{}^{}f(x)dx} = \lim_{n \to \infty} \int_{}^{} f_n(x)dx \leq + \infty.
    \end{equation}
  \end{enumerate}
\end{prop}

Au lieu d'intégrer \(f\) sur tout \(\mathbb{R}\), on peut l'intégrer seulement sur la partie où elle est mesurable en posant :

\[\int_{A}^{}f(x)dx = \int_{}^{}f(x)\mathds{1}_{A}(x)dx.\]

\begin{thm}
  On peut calculer l'intégrale \(\int_{A} f(x)dx \) de toute fonction mesurable positive par :

  \[\int_{A} f(x)dx = \sup \sum_{i=0}^{n-1} (t _{i+1}-t_i) \mu(A \cap \{ f_i \geq t_i \}) \]

  où le \(\sup\) est pris sur toutes les subdivisions finies sur l'axe des \(y\), \(t_0 \less t_1 \less \dots \less t_n, n \in \mathbb{N}\) et dont le pas tend vers 0.
\end{thm}

\begin{figure}[h!]
  \centering
  \includegraphics[scale=1]{figures/riem.png}
  \caption{Subdivisions et sommes de Riemann}
  \label{subdiv_riemann}
\end{figure}

\begin{prop}
  Si \(f\) est à valeurs positives, alors

  \[\int_{}^{}f(x)dx = 0 \text{ si et seulement si } f=0 \text{ p.p. }   \]
\end{prop}

\begin{proof}
  Posons \(A = \{ x \in X, f(x) \neq 0 \}\). Alors \(f(x) \leq \lim_{n \to \infty} n \mathds{1}_{A}(x)\) si \(\mu(A) = 0\). On obtient par \ref{Beppo-Levi} :

  \[\int_{}^{}f(x)dx \lim_{n \to \infty} \int_{A}^{}dx = 0.  \]

  Réciproquement, si \(\int_{}^{}f(x)dx =0 \), alors on remarque que \(\mathds{1}_{A}(x) \leq \lim_{n \to \infty} n f_n(x)\) et on a encore par \ref{Beppo-Levi} :

  \[\mu(A) = \int_{A}^{}\mathds{1}_{A}(x)dx \leq \lim_{n \to \infty} n \int_{}^{}f(x)dx = 0.\]
\end{proof}

\marginnote{25-10-2023}

\begin{thm}[{\fontencoding{U}\fontfamily{futs}\selectfont\char 66\relax} De convergence dominée ou de Lebesgue]
  Soit \((f_n) \in \mathscr{L}^{1}(\Omega)\) une suite de fonctions (avec \(\Omega \subset \mathbb{R}^d\) un ouvert). Pour tout \(n \in \mathbb{N}, f_n\) est définie presque partout sauf sur \(E = \left(\bigcup _{n \in \mathbb{N}} E_n\right)\) de mesure nulle. S'il existe \(h \in \mathscr{L}^{1}(\Omega)\)
  telle que \(\left\lvert f_n \right\rvert \leq h\) et si \(f_n \longrightarrow f \) presque partout sur \(\Omega\), alors

  \[\lim_{n \to \infty}  \int f_n(x)dx = \int f(x)dx,\]

  ce qui équivaut à dire que \(\left\Vert f_n - f \right\Vert _{\mathscr{L}^{1}}\longrightarrow 0\).
\end{thm}

\begin{remark}
  Si \(f_n\) est continue sur \(K \subset \Omega\) et \(f_n \longrightarrow f\) quand \(n \longrightarrow \infty\), alors

  \[\lim_{n \to \infty} \int f_n = \int \lim_{n \to \infty} f_n.\]
\end{remark}

\section{Espaces \(L ^{p} \ (1 \leq  p \leq +\infty)\) comme espaces de Banach}

On définit l'espace \(\mathscr{L}^{1} \longrightarrow L ^{1}\). On dit que \(f \sim g\) si et seulement si \(f = g\) presque partout (i. e. \(\mu (\{ x : f(x)\neq g(x) \}) = 0\)). C'est une relation d'équivalence. On définit l'espace quotient \(L ^{1} = \mathscr{L}^{1} / \sim\). On montre que \(L ^{1}\) est complet.

On note \(\dot{f} \in L ^{1}\), \(\dot{f}\) est un représentant de la classe de \(f\). Ainsi on peut définir la transformée de Fourier d'une fonction de \(L ^{1}\) par :

\[\widehat{f}(\xi) = \int_{}^{} e^{-ix \xi}f(x)dx,\]

mais la même écriture n'a pas de sens dans \(L ^{2}\).

\subsection{Les espaces \(L ^{p}, 1 \leq p \less \infty\)}

\begin{thm}
  L'espace \(L ^{1}\) est complet, muni de la norme

  \[\left\Vert f \right\Vert _{1} = \int_{}^{} \left\lvert f(x) \right\rvert dx.\]

  Cela veut dire que c'est un espace de Banach.
\end{thm}

\begin{proof}
  Soit \(f_n\) une suite de Cauchy dans \(L ^{1}\). On peut lui associer une série

  \[f_n = \sum_{i=0}^{n-1} f_i (f _{i+1} -f_i).\]

  Montrons que \(f_n\) converge. Il suffit de montrer qu'il existe une sous-suite \(f _{n_k}\) qui converge. On peut toujours écrire :

  \[f_n - f = \underbrace{f_n - f _{n_k}}_{\less \varepsilon}+ \underbrace{f_n - f _{n_k}}_{\less \varepsilon}.\]

  Donc \[\forall \varepsilon \bg 0, \exists N \in \mathbb{N}, p \geq  N, q \geq 0, \left\Vert f _{p+q} - f _{p} \right\Vert _{L ^{1}} \less \varepsilon.\]

  A extraction près d'une sous-suite, on peut supposer que

  \[\left\Vert f _{n+1}-f_n \right\Vert _{L ^{1}} \less \varepsilon.\]

  On pose \(g _{m}(x) = \displaystyle \sum_{n=1}^{m-1} \left\lvert f _{n+1}(x) - f_n(x) \right\rvert \). Pour tout \(x\), la suite \((g_m(x))_{m}\) est croissante (car on rajoute un terme positif) et elle est définie presque partout. Par Beppo-Levi, on a

  \begin{gather*}
    \int g(x) dx = \int \lim_{m \to \infty} g_m = \lim_{m \to \infty} \int g_m(x)dx.
  \end{gather*}

  Or on a

  \begin{gather*}
    \int g_m(x)dx = \sum_{n=1}^{m-1} \int \left\lvert f _{n+1}(x)-f_n(x) \right\rvert dx.
  \end{gather*}

  C'est une série absolument convergente, car \(\left\Vert f _{n+1}-f_n \right\Vert \less 2 ^{-n}\). On a alors

  \[\int g(x)dx \less +\infty\]

  et en particulier \(g(x) \less +\infty\) presque partout.

  Donc \(f_m(x) = f_1(x) + \displaystyle \sum_{n=1}^{m-1} f _{n+1}(x) - f_n(x)\) définit une série numérique absolument convergente, donc \(f_m(x) \underset{\text{p.p.}}{\longrightarrow} f(x)\). Il reste à montrer que \(f \in L ^{1}\). On a :

  \begin{gather*}
    \left\lvert f_m(x) \right\rvert \leq \left\lvert f_1(x) \right\rvert + g_m(x) \leq \left\lvert f_1(x) \right\rvert + g(x) = h(x) \in L ^{1}.
  \end{gather*}

  Par Lebesgue, on a \(\lim \int f_m = \int f\), avec \(f \in L ^{1}\). On a donc \(\left\Vert f_m - f \right\Vert _{L ^{1}} \longrightarrow 0\).
\end{proof}

\begin{thm}
  \(L ^{p}\) est aussi un Banach et ce pour tout \(p \in [1, \infty)\).
\end{thm}

\begin{proof}
  On utilise l'inégalité de Minkowski :

  \begin{equation}
    \left\Vert f+g \right\Vert _{L ^{p}} \leq \left\Vert f \right\Vert _{L ^{p}} + \left\Vert g \right\Vert _{L ^{p}}.
  \end{equation}

  La démonstration est la même que pour \(L ^{1}\).
\end{proof}

\begin{corollary}
  Si \(f_n \longrightarrow f\) dans \(L ^{p}\), alors il existe une sous-suite \(f _{n_k}\) qui converge presque partout vers \(f\).
\end{corollary}

\begin{proof}
  Comme avant, on peut supposer que \(\left\Vert f _{n+1} - f_n \right\Vert \less 2^{-n}\). On pose

  \[f_m(x) = f_1(x) + \sum_{n=1}^{m-1} f _{n+1}(x)-f_n(x).\]

  C'est une série normalement convergente dans \(L ^{p}\). Par le raisonnement précédent, on détermine que \(f_m(x) \underset{\text{p.p.}}{\longrightarrow} f\).
\end{proof}

\begin{thm}
  Soit \(\mathcal{K}(\Omega)\), l'espace de fonctions continues à support compact dans \(\Omega\). Alors \(\mathcal{K}(\Omega)\) est dense dans \(L ^{p}(\Omega)\) pour tout \(p \in [1, \infty)\).
\end{thm}

\begin{proof}
  Brézis, théorème 4.12.
\end{proof}

\subsection{Espace \(L ^{\infty}\)}

\begin{thm}
  \(L ^{\infty}(\Omega)\) est un espace de Banach.
\end{thm}

\begin{definition}
  On définit la norme dans l'espace \(L ^{\infty}\) de la façon suivante :

  \[\left\Vert f \right\Vert _{\infty} = \operatorname{supess} \left\lvert f \right\rvert.\]

  Pour tout \(C \bg \operatorname{supess}(f)\), on a \(\mu(\{ x : f(x) \bg C \})=0\). Si \(f\) est continue, on a \(\operatorname{supess} f = \sup f\).
\end{definition}

\begin{proof}
  Soit \(f_n\) une suite de Cauchy dans \(L ^{\infty}(\Omega)\). On prend \(\varepsilon = \displaystyle \frac{1}{k}\). Pour tout \(k\), il existe \(N(k)\) tel que \(\forall n, m \geq N(k)\), on a \(\left\Vert f_n - f_m \right\Vert _{\infty} \less \displaystyle \frac{1}{k}\).

  Montrons que \(f_n\) converge vers \(f \in L ^{\infty}\). Donc il existe \(E_k\) négligeable tel que \(\left\lvert f_n(x) - f_m(x) \right\rvert \less \displaystyle\frac{1}{k}\) pour tout \(x \notin E_k\). On a que \(E = \displaystyle \bigcup _{k \in \mathbb{N}} E_k\) est de mesure nulle. Pour tout \(x \in E\), la suite numérique \(f_n(x)\) est de Cauchy, donc elle converge vers \(f(x)\) (partout). On a donc

  \[\left\lvert f_n(x) - f_m(x) \right\rvert \less \frac{1}{k},\]

  donc \[\left\lvert f_n(x) - f(x) \right\rvert \less \frac{1}{k},\]

  donc \(\left\Vert f_n - f \right\Vert _{\infty} \less \frac{1}{k}\) avec \(f \in L ^{\infty}\).
\end{proof}

\subsection{Espace \(L ^2\)}

\begin{thm}\label{l2-hilbert}
  \(\mathcal{H} = L ^2\) est un espace de Hilbert muni du produit scalaire suivant :

  \[(u \mid v) = \int_{\Omega} f(x) \overline{g(x)}dx.\]
\end{thm}

L'espace \(L ^2\) est engendré par une base orthonormée (\emph{base hilbertienne}), i.e. il existe une suite \((e_j)_j\) telle que, pour tout \(f \in L ^2\), on a

\[f = \sum_{j} f_j e_j, f_j \in \mathbb{C}.\]

\begin{prop}[Egalité de Parseval]
  On a pour tout \(f \in L ^2\) :

  \[\left\Vert f \right\Vert _{2}^2 = \sum_{j}^{} \left\lvert f_j \right\rvert ^2.\]
\end{prop}

\section{Espaces \(L ^{p}\) comme ELC}

\subsection{\(L ^{1}(\Omega)\), avec \(\Omega\) ouvert de \(\mathbb{R}^d\)}

C'est un espace de Banach muni de la norme \(\left\Vert f \right\Vert _{L ^{1}}\). C'est aussi un espace localement convexe muni de la famille de semi-normes \((\rho _{a,r}) _{a \in \Omega, r \bg 0}\) définies par

\[\rho _{a,r}(f) = \int_{B(a,r)}^{}\left\lvert f(x) \right\rvert dx.\]

\begin{prop}
  \((E, (\rho _{a,r}))\) est séparé.
\end{prop}

\begin{proof}
  Si \(f \neq g\) presque partout, alors

  \begin{gather*}
    \rho _{a,r}(f-g) = \int_{B(a,r)} \left\lvert f(x) -g(x)\right\rvert dx \neq 0.
  \end{gather*}

  Donc il existe \(E\) de mesure strictement positive tel que \(\forall x \in E, g(x) \neq f(x)\). Or \(E\) est partout dense dans \(\Omega\) pour la mesure de Lebesgue. On a alors

  \[\int_{B(a,r)} \left\lvert f-g \right\rvert = \int_{E \cap B(a,r)} \left\lvert f-g \right\rvert \bg 0.\]
\end{proof}

\subsection{\(E = L ^{\infty}\)}

On le munit de la famille de semi-normes

\[\rho _{a,r}(f) = \operatorname{supess}_{B(a,r)}(\left\lvert f \right\rvert).\]

\begin{prop}
  \(E = L ^{\infty}(\Omega)\) est séparé, mais non séparable.
\end{prop}

\subsection{Espaces duaux de \(L ^{p}\)}

\begin{thm}[De Riesz]
  On a \((L ^2)' = L ^2\) (l'espace dual de \(L ^2\) est lui-même), i. e. toute forme linéaire \(l \in (L ^2)'\) s'écrit comme \(\langle l,u \rangle  = (v \mid u)\) où \(v \in L ^2\).
\end{thm}

\begin{thm}[Riesz-Fischer]\label{Riesz-Fischer}
  Le dual de \(L ^{1}\) est \(L ^{\infty}\).
\end{thm}

\begin{proof}
  Brézis, p. 63.
\end{proof}

\begin{remark}
  Par contre on n'a pas \((L ^{\infty})' \neq L ^{1}\).
\end{remark}

\marginnote{27-10-2023}

\begin{corollary}\label{int_nulle}
  Soit \(f \in L ^{p}\). Si pout tout \(\varphi \in K(\Omega), \int_{}^{}f(x) \varphi(x)dx = 0\), alors \(f = 0\) presque partout.
\end{corollary}

\begin{thm}
  L'espace \(L^{1}\) est séparable. Plus généralement, \(L ^{p}\) est séparable pour tout \(1 \leq p \less +\infty\).
\end{thm}

\begin{proof}
  On prend un représentant \(\dot{f}\) de \(f \in L ^{1}\). Si \(f\) est positive, on a

  \begin{gather*}
    \int_{A}^{} f(x)dx = \sup_{t_0 \leq t_1 \leq \dots \leq t_n} (t _{i+1} - t_i) \mu(A \cap \{ x : f(x) \leq t_i \}),
  \end{gather*}

  où \(\mu\) est la mesure de Lebesgue. On a alors

  \begin{gather*}
    f = \lim \sum_{i}^{} c_i \mathds{1}_{\{ f \leq t_i \}} = \lim \sum_{t}^{} (t _{i+1}-t_i) \mathds{1}_{B_i}(t).
  \end{gather*}

  On peut prendre \(t_i \in \mathbb{Q}\). Alors la famille

  \begin{gather*}
    \sum_{i}^{} (t _{i+1} - t_i) \mathds{1}_{B_i}(t)
  \end{gather*}

  est partout dense dans \(L ^{1}\), avec \(B_i\) des boréliens de \(\Omega\). Ceci achève la démonstration.
\end{proof}

\begin{thm}
  L'espace \(L ^{1}(\Omega)\) n'est pas réfléxif.
\end{thm}

\begin{proof}
  On raisonne par l'absurde. On pose \(E = L ^{1}\) et on suppose qu'il est réfléxif. On a alors, par le théorème \ref{Riesz-Fischer}, \(E' = L ^{\infty}\).

  Soit \(B_E\) la boule unité de \(E\) (pour la topologie de Banach, mais aussi valable pour \(\sigma(E, E')\)). Comme \(L ^{1}\) est séparable, la boule unité \(B_E\) pour la topologie faible est compacte. Donc de toute suite de \(B_e\) on peut extraire une suite convergente (pour la topologie faible). On prend

  \[f_n(x) = \frac{1}{\left\lvert B(a, \frac{1}{n}) \right\rvert} \mathds{1}_{B(a, \frac{1}{n})}.\]

  On a \(\int f_n = 1\) pour \(f_n \in B_E\). On entrait \(f _{n_k}\) convergeant vers \(f \in B_E\). Alors pour tout \(\phi \in L ^{\infty} = E'\), on a \(\int f _{n_k} \phi \longrightarrow \int f \phi\). On choisit \(\phi \in K(\Omega \setminus \{ a \})\). Donc

  \[\int f _{n_k}(x) \phi(x) dx =0.\]

  Par le corollaire \ref{int_nulle}, \(f = 0\) p. p. dans \(\Omega \setminus \{ a \}\). Donc \(f = 0\) p. p. dans \(\Omega\), car \(\mu(\{ a \}) = 0\). Cela aboutit a une contradiction, car pour \(\phi = \mathds{1}_{\Omega} \in L ^{\infty}\), on a

  \[0 = \int f \phi = \int_{}^{} \frac{1}{\left\lvert B(a, \frac{1}{n}) \right\rvert} \mathds{1}_{B(a, \frac{1}{n})} = 1.\]
\end{proof}

\begin{remark}
  \(E\) est réfléxif et séparable si et seulement si \(E'\) est réfléxif et séparable. Comme \(E = L ^{1}\) n'est pas réfléxif, \(E' = L ^{\infty}\) n'est pas réfléxif. En fait il n'est ni réfléxif ni séparable.
\end{remark}

\begin{prop}
  \(L ^{\infty}\) n'est pas séparable.
\end{prop}

\begin{lemma}
  Soit \(E\) un espace de Banach. S'il existe \((\mathcal{O}_i)_{i \in I}\), \(I\) non dénombrable, \(\mathcal{O}_i\) ouverts deux-à-deux disjoints, alors \(E\) n'est pas séparable.
\end{lemma}

\begin{proof}
  Une fois de plus on raisonne par l'absurde. Soit \((u_n)\) une suite partout dense dans \(E\) telle que \(\forall i \in I, (u_n) _{n \in \mathbb{N}} \cap \mathcal{O}_i \neq \emptyset\) (on utilise l'axiome du choix \(u_n \in \mathcal{O}_i\)). Comme les \(\mathcal{O}_i\) sont disjoints, \(i \longmapsto n(i)\) est injective, donc \(I\) est dénombrable, ce qui aboutit à une contradiction.
\end{proof}

\begin{thm}
  Soit \(\Omega\) un ouvert de \(\mathbb{R}^d\). Alors

  \begin{enumerate}
    \item \(L ^{\infty}(\Omega)\) n'est pas séparable ;
    \item \(L ^{\infty}(\Omega)\) n'est pas réfléxif ;
    \item La boule unité de \(L ^{\infty}(\Omega)\) est métrisable est compacte pour \(\ast(L ^{\infty}, L ^{1})\).
  \end{enumerate}
\end{thm}

\section{Exemple fondamental : convergence d'une suite de \(L ^{1}\) vers la mesure de Dirac}

\begin{thm}
  Il existe \(f_n \in L ^{1}\) telle que \(\forall \phi \in K(\Omega)\),

  \[\int f_n(x) \phi(x)dx \longrightarrow \phi(x_0).\]
\end{thm}

\chapter{Fonctions troncature et partition de l'unité : cas continu}\marginnote{07-11-2023}

En théorie de distributions, on déduit souvent une ``propriété globale'' à partir d'une ``propriété locale''. Ceci se fait par une sorte de ``copié-collé'' par des partitions de l'unité.

\section{Les fonctions troncature}

\begin{prop}\label{11}
  Soit \((X,d)\) un espace métrique. Soient \(F\) et \(G\) deux fermés disjoints de \(X\). Alors il existe une fonction continue \(\chi \in \mathcal{C}^0(X), 0 \leq \chi \leq 1\) et \(\chi \equiv 0\) près de \(F\) et \(\chi \equiv 1\) près de \(G\).
\end{prop}

\begin{proof}
  En deux étapes.

  \begin{enumerate}
    \item On construit \(\chi_1 \equiv 0\) sur \(F\) et \(\chi_1 \equiv 1\) sur \(G\). On définit \(\chi : X \longrightarrow [0,1]\), avec \(\chi_1(x) = \displaystyle \frac{d(x,F)}{d(x,F)+ d(x,G)}\). La continuité de \(\chi\) résulte du fait que \(x \longmapsto d(x,F)\) est continue. On a de plus \(\chi_1 \bigm|_F = 0\) et \(\chi_1 \bigm|_G = 1\).

    Considérons \(F_1 = \chi ^{-1}\left(\left[0,\frac{1}{3}\right]\right) \supset F\) et \(G_1 = \chi_1 ^{-1}\left(\left[\frac{2}{3}, 1\right]\right) \supset G\). On a \(F_1 \cap G_1 = \emptyset\). On applique la construction précédente à \(F_1\) et \(G_1\) qui sont des voisinages \textbf{fermés} de \(F\) et de \(G\).
  \end{enumerate}
\end{proof}

\begin{lemma}\label{12}
  Soit \((X,d)\) un espace métrique et \(K \subset X\) un compact. Soit \((U_j)_{1 \leq j \leq N}\) un recouvrement ouvert de \(K\). Alors il existe \(K_j \subset U_j \subset X\) compact, \(1 \leq j \leq N\) tel que \(K \subset \displaystyle \bigcup _{j=1}^{N} K_j\).
\end{lemma}

\begin{proof}
  On remarque que \(K \subset \displaystyle\bigcup _{x \in K} B_x\) où \(B_x\) est une boule ouverte de centre \(x\). Cela entraîne que \(K \subset \displaystyle \bigcup _{k=1}^{p} B _{x_k}\).

  On considère \(A_j = \{ l \in \{ 1, \dots, p \}, \widetilde{B _{x_l}} \subset U_j\}\) où \(\widetilde{B _{x_l}}\) est une boule fermée de même centre et de même rayon que \(B _{x_l}\).

  \begin{remark}
    \(\overline{B_{x_l}} \subset \widetilde{B_{x_l}}\), mais l'inclusion est stricte en général.
  \end{remark}

  Si \((X,d) = (\mathbb{R}^n, \left\Vert \cdot \right\Vert )\), on a \(\overline{B _{x_l}} = \widetilde{B _{x_l}}\). On pose \(K_j = K \cap \left(\displaystyle\bigcup_{l \in A_j}\widetilde{B _{x_l}}\right)\), alors \(K_j\) est compact (tout fermé dans un espace séparé est compact).
\end{proof}

\begin{thm}
  Soit \(K \subset \bigcup U_j\) compact. Alors il existe \(\varphi_j \in \mathcal{K}(X)\) tels que \(\operatorname{supp}\varphi_j \subset U_j\) et \(\displaystyle \sum_{j=1}^{n} \varphi_j = 1\) \textbf{près} de \(K\), avec \(0 \leq \varphi_j \leq 1\). On dit que les \(\varphi_j\) forment une partition de l'unité subordonnée au recouvrement de \(K\) par un nombre fini d'ouverts \(U_j\).
\end{thm}

\begin{proof}
  Soient \(K_j\) des compacts comme dans le lemme \ref{12}. Pour chaque \(j \in \{1, \dots, N \}\), on peut trouver une fonction continue \(\psi_j : X \longrightarrow [0,1]\) égale à 1 près de \(K_j\) et égale à 0 sur \(U_j ^{C}\). On pose

  \[V = \left\{ x \in X, \sum_{j=1}^{N} \psi_j(x) \bg 0 \right\}\]

  ouvert de \(X\), donc \(V ^{C}\) est fermé. On applique la proposition \ref{11} à \(K\) et \(V^{C}\). Il existe donc \(\psi_0\) continue avec \(\psi_0 \equiv 0\) près de \(K\) et \(\psi_0 \equiv 1\) près de \(U ^{C}\), car \(K \subset U\).

  On pose

  \[\varphi_j(x) = \frac{\psi_j(x)}{\sum_{k=0}^{N} \psi_k(x)}.\]

  La fonction \(\psi_j\) est dans \(\mathcal{K}(X)\) et \(\displaystyle\sum_{j=1}^{N}\varphi_j = \displaystyle \frac{\sum_{j=1}^{N} \psi_j }{ \psi_0 + \sum_{j=1}^{N}\psi_j } \equiv 1\) près de \(K\), car \(\psi_0 \equiv 0\) près de \(K\).
\end{proof}

\paragraph{Application : recollement d'une famille de \(L ^{1}_{\operatorname{loc}}(\mathbb{R}^d)\)}

Ce sont l'ensemble de fonctions intégrables seulement sur un compact. On a \(L ^{p}(X) \subset L ^{1}_{\operatorname{loc}}(X)\). On a

\[\int_{K}^{} \left\lvert f \right\rvert  = \int \mathds{1}_{K}(x) \left\lvert f(x) \right\rvert dx = \left\Vert \mathds{1}_{K} \right\Vert _{q} \left\Vert f \right\Vert _{p} = \left\lvert K \right\rvert ^{\frac{1}{q}} \left\Vert f \right\Vert _{p}.\]

\begin{prop}
  Soient \(U_j \subset \mathbb{R}^d\) des ouverts, \(\Omega _{j \in I} U_j, f_j \in L ^{1}_{\operatorname{loc}}(U_j)\) avec \(f_i = f_j\) sur \(U_i = U_j\). Alors il existe \(f \in L ^{1}_{\operatorname{loc}}(\Omega)\) telle que \(f = f_j\) sur \(U_j\).
\end{prop}

\begin{proof}
  On va montrer qu'il existe \(f \in L ^{1}_{\operatorname{loc}}(\Omega)\) telle que \(\forall g \in L ^{\infty}_{\operatorname{comp}}(\Omega)\),

  \[\langle f,g \rangle = \int_{}^{} f(x)g(x)dx.\]
\end{proof}

\chapter{Fonctions différentiables} \marginnote{10-11-2023}

\section{Rappels de calcul différentiel}

\begin{definition}
  Soit \(\Omega \in \mathbb{R}\) ouvert. On dit que \(f : \Omega \longrightarrow \mathbb{R}/ \mathbb{C}\) est de classe \(\mathcal{C}^1\) si et seulement si \(f\) admet des dérivées partielles continues.

  Si \(x_0 \in \Omega\), \(f'(x_0)\) est la différentielle de \(f\) en \(x_0\) telle que

  \[\langle f'(x_0), y \rangle = \sum_{i=1}^{d} \frac{\partial f }{\partial x_i}(x_0) y_j.\]

  \(f\) est de classe \(\mathcal{C}^2\) si et seulement si \(x \longmapsto \displaystyle \frac{\partial f }{\partial x_j}(x)\) est de classe \(\mathcal{C}^1\) pour tout \(j\). La matrice hessienne de \(f\) en \(x_0\) est

  \[\operatorname{Hess}(f) = \left(\frac{\partial ^2 f(x_0)}{\partial x_i \partial x_j} \right)_{1 \leq i,j \leq d}.\]

  Par le théorème de Schwarz, elle est symétrique.

  Par récurrence, on définit les fontions \(\mathcal{C}^{\infty}(\Omega)\) et

  \[\mathcal{C}^{\infty}(\Omega) = \bigcap _{n \in \mathbb{N}}\mathcal{C}^{\infty}(\Omega).\]

  \(\forall n \in \mathbb{N}\), \(\mathcal{C}^{\infty}(\Omega)\) est une algèbre.
\end{definition}

\begin{prop}[Composition]
  Soient \(\Omega_1 \in \mathbb{R}^{d_1}, \Omega_2 \in \mathbb{R}^{d_2}\) des ouverts de \(\mathbb{R}^d\) et soit \(\Phi : \Omega_1 \longrightarrow \Omega_2\) de classe \(\mathcal{C}^{\infty}\). Alors \(\forall f \in \mathcal{C}^{\infty}(\Omega_2), f \circ \Phi \in \mathcal{C}^{\infty}(\Omega_1)\) et on a

  \[(f \circ \Phi)'(x_0) = \underbrace{f'(\Phi(x_0))}_{\in \mathcal{M}_{1 \times d_2}} \cdot \underbrace{\Phi'(x_0)}_{\in \mathcal{M}_{d_2 \times d_1}}.\]
\end{prop}

\begin{exo}
  Calculer \(((f \circ \Phi)')'(x)\).
\end{exo}

\paragraph{Formule de Taylor avec reste intégral}

  Pour \(d=1\), on a

  \[f(x) = f(0) + \int_{0}^{x}f'(y)dy = f(0)+ \left(\int_{0}^{1}f'(tx)dt \right)x. \]


\begin{remark}
  Si \(f \in C_0 ^{1}(\mathbb{R})\), la fonction \(g = \int_{0}^{1}f'(tx)dx\) n'est plus dans \(C_0 ^{1}(\mathbb{R})\).
\end{remark}

\begin{prop}
  Soit \(\Omega \subset \mathbb{R}^d, a \in \Omega\) et \(f \in \mathcal{C}^{\infty}(\Omega)\). Alors on peut écrire :

  \[f(x) = \sum_{\left\lvert \alpha \right\rvert \less m} \frac{1}{\alpha!}\partial ^{\alpha}f(a)(x-a)^{\alpha}+ m \sum_{\left\lvert \alpha \right\rvert = m} \frac{(x- \alpha)^2}{\alpha!} \int_{0}^{1}(1-t)^{m-1}\partial ^{\alpha}f(a+t(x-a))dt.\]

  On a \(\alpha \in \mathbb{N} ^{d}, \left\lvert \alpha \right\rvert = \alpha_1 + \dots + \alpha_d\) et \(\alpha! = \alpha_1 ! \dots \alpha_d !\) et \(\partial ^{\alpha} = \displaystyle \frac{\partial ^{\alpha_1} }{\partial x_1 ^{\alpha_1} } \dots \displaystyle \frac{\partial ^{\alpha_d} }{\partial x_d^{\alpha_d} }\).
\end{prop}

\begin{remark}[Formule d'Hadamard]
  Pour \(m=1\) et \(\alpha = (0, \dots, 0, \underbrace{1}_{j}, 0, \dots, 0)\), on a

  \[f(x) = \sum_{j=0}^{d}(x_j - a_j) \int_{0}^{1} \frac{\partial f }{\partial x_j}(a+t(x-a))dt.\]
\end{remark}

\section{Classification des \(\mathcal{C}^{m}(\Omega)\), régularité, support}

Soit \(K \subset \Omega \subset \mathbb{R}^d\) compact. On définit

\[C ^{m}_{K}(\Omega) = \{ f \in \mathcal{C}^{m}(\Omega), \ \operatorname{supp}(f) \subset K\}.\]

On rappelle que \(\operatorname{supp}(f) = \operatorname{adh} \{ x \mid f(x)=0 \}\) (l'ensemble des valeurs où la fonction ne s'annulle pas).

Les espaces localement convexes sont munis d'une famille de semi-normes qui leur confère une structure des espaces de Fréchet.

\begin{exemple}

  \

  \begin{itemize}
    \item [$\star$] Pour \(\mathcal{C}^m(\Omega)\), on a

    \[\rho _{n,\alpha:\left\lvert \alpha \right\rvert \leq m} = \sup_{K_n} \left\lvert \partial ^{\alpha}f(x) \right\rvert\]

    où \(K_n\) est une suite exhaustive de compacts avec \(K_n \subset K _{n+1}, \displaystyle \bigcup _{n \in \mathbb{N}}K_n = \Omega\).

    \item [$\star$] Pour \(C ^{m}_{h}(\Omega), f ^{(1)}, \dots, f ^{(m)}\) bornées, on a

    \[\left\Vert f \right\Vert = \sum_{\left\lvert \alpha \right\rvert \less m} \sup_{\Omega} \left\lvert \partial ^{\alpha}f(x) \right\rvert.\]

    C'est une norme sur un espace de Banach.

    \item[$\star$] On considère \(C _{0}^{\infty}(\Omega) = \bigcup _{K} C _{K}^{\infty}(\Omega)\) espace localement convexe.

    \begin{remark}
      \[\sup_{\nu \in \mathbb{N}}\left(\sup_{\left\lvert x \right\rvert \bg \nu, \left\lvert \alpha \right\rvert \leq m _{\nu}}\frac{\left\lvert \partial ^{\alpha}\varphi(x) \right\rvert}{\varepsilon_\nu}\right).\]

      \(\varepsilon_\nu\) est une suite de réels et elle tend vers 0.
    \end{remark}

    ...
  \end{itemize}
\end{exemple}

On prend \(\mathscr{D}(\Omega) \neq \{ 0 \}\).

\begin{lemma}\label{lemme21}
  Soit \(B(a,r)\subset \Omega, a \in \Omega, r \bg 0\). Soit \(\Phi_a : \mathbb{R}^d \longrightarrow \mathbb{R} _{+}\) telle que

  \[\Phi_a (x) = \begin{cases}
    0 \text{ si } x \notin B(a,r) \\
    \exp \left(- \frac{1}{r^2- \left\lvert x-a \right\rvert ^2}\right) \text{ si } x \in B(a,r).
  \end{cases}\]

  Alors \(\Phi_a \in C ^{\infty}_{\overline{B(a,r)}}(\Omega)\). Elle est dans \(\mathscr{D}(\mathbb{R}^{d})\).
\end{lemma}

\begin{proof}
  Soit \(\psi : \mathbb{R} \longrightarrow [0,\infty)\) telle que

  \[\psi(t) = \begin{cases}
    0 \text{ si } t \leq 0 \\
    e^{-\frac{1}{t}} \text{ sinon}.
  \end{cases}\]

  Il est connu que \(\psi \in \mathcal{C}^{\infty}(\mathbb{R})\).

  On construit

  \[\Phi(x) = \psi(r ^2 - \left\lvert x-a \right\rvert ^2).\]

  Alors \(\Phi_a \in \mathcal{C}^{\infty}(\Omega)\) et \(\operatorname{supp} \Phi_a \subset B(a,r)\) par construction.
\end{proof}

\begin{lemma}
  Il existe une fonction \(\Psi \in \mathcal{C}^{\infty}(\mathbb{R})\) croissante telle que

  \[\Psi(t) = \begin{cases}
    0 \text{ si } t \leq 0 \\
    1 \text{ si } t \geq 1.
  \end{cases}\]
\end{lemma}

\begin{proof}
  On pose \(g(t) = \psi(t)\psi(1-t)\in C ^{\infty}_{[0,1]}(\mathbb{R})\) et on pose

  \[\Psi(t) = \frac{\int_{-\infty}^{t}g(t) dt}{\int_{0}^{1}g(s) ds}.\]
\end{proof}

\section{Partitions de l'unité différentiables}

\begin{lemma}
  Soit \(\Omega \subset \mathbb{R}^d\) ouvert et \(K \subset \Omega\) compact. Alors il existe une fonction \(f : \Omega \longrightarrow [0,1], f \in \mathscr{D}(\mathbb{R})\) telle que \(f \equiv 1\) au voisinage de \(K\).
\end{lemma}

\begin{proof}

  \begin{figure}[h!]
    \centering
    \includegraphics[scale=0.3]{figures/part_unit_diff21.png}
    \caption{Recouvrement de \(K\)}
    \label{}
  \end{figure}

  On applique le lemme \ref{lemme21}. On pose

  \[\Phi_x(y) = \exp \left(\frac{1}{r ^2- \left\lvert y-x \right\rvert ^2}\right) \times 2 \exp \left(\frac{1}{r ^2}\right).\]

  On a \(\Phi_x(x) = 2\).

  \(x \in V_x = \{ y : \Phi_x(y) \bg 1 \}\) un ouvert. Les \(V_x\) recouvrent \(K\). On peut extraire un sous-recouvrement fini \(V _{x_1}, \dots, V _{x_N}\). On pose

  \[h = \sum_{j=1}^{N} \Phi _{x_j}, h \in C_0 ^{\infty}(\Omega), h \geq 1 \text{ près de } K.\]

  Alors \(f = \psi \circ h\) répond à la question.
\end{proof}

\begin{proof}[Démonstration par régularisation des convolutions]
  \begin{figure}[h!]
    \centering
    \includegraphics[scale=0.3]{figures/reg_conv.png}
    \caption{On prend \(K\) et \(U\) définis ainsi.}
    \label{}
  \end{figure}

  Soit \(g = \mathds{1}_{U}\) et \(\varepsilon \less \frac{\delta}{2}\). Posons \(\Phi_1 \in C _{0}^{\infty}(\mathbb{R}^d)\). Alors \(\operatorname{supp}\Phi_1 \subset B(0,1)\).

  Posons \(\Phi _{\varepsilon}(x) = \frac{1}{\varepsilon ^{d}}\Phi_1 \left(\frac{x}{\varepsilon}\right)\). On a \(\operatorname{supp}(\Phi _{\varepsilon}) \subset B(0,\varepsilon)\).

  On a \[f(x) = f _{\varepsilon}(x) = \int \mathds{1}_{U}(y)\Phi _{\varepsilon}(x-y)dy = \mathds{1}_{U} \star \Phi _{\varepsilon}.\]

  On remarque que \(f _{\varepsilon} \in \mathcal{C}^{\infty}(\mathbb{R}^{d})\). La convolution \(f \star g\) hérite de la meilleure régularité. On  a \(\operatorname{supp}(f \star g) \subset \operatorname{supp}(f) + \operatorname{supp}(g)\).

  Donc \(\operatorname{supp}(f _{\varepsilon}) \subset U + B(0,\varepsilon) \subset K + B(0,\delta)+ B(0,\varepsilon)\subset K+ B \left(0, \frac{3 d}{2}\right)\).

  Montrons que \(\operatorname{supp}f _{\varepsilon}\subset K + B \left(0, \frac{\delta}{2}\right)\). Alors

  \begin{gather*}
    f _{\varepsilon}(x) = \int \mathds{1}_{U}(y) \Phi _{\varepsilon}(x-y)dy = 0 \text{ pour } d(x,U) \bg \varepsilon.
  \end{gather*}

  Montrons que \(f _{\varepsilon}(x)=1\) pour \(d(x,U) \less \frac{\varepsilon}{2}\). Cela implique que

  \begin{gather*}
    \int \mathds{1}_{U}(y)\Phi _{\varepsilon}(x,y)dy = \int \Phi _{\varepsilon}(x-y)dy = \int \Phi _{\varepsilon}(z)dz = 1.
  \end{gather*}

  Montrons que \(f _{\varepsilon}(x) \leq 1\). C'est vrai car \(\mathds{1}_{U}(y)\in [0,1]\) et on a

  \[\int \mathds{1}_{U}(y)\Phi _{\varepsilon}(x-y)dy \leq \int \Phi _{\varepsilon}(x-y)dy=1.\]
\end{proof}

Ce lemme aboutit au théorème suivant.

\begin{thm}
  Soit \(K \in \mathbb{R}^d\) un compact recouvert par une union fini d'ouverts \((U_j)_{1 \leq j \leq n}\). Alors il existe \(\varphi_j \in \mathscr{D}(\mathbb{R}^d), j = 1, \dots, n, \operatorname{supp}\varphi_j \subset U_j\) et \(\displaystyle \sum_{j=1}^{n} \varphi_j = 1 \) près de \(K\).
\end{thm}

\end{document}
