\documentclass[french]{book}
\usepackage[utf8x]{inputenc}
\usepackage[T1]{fontenc}
\usepackage{babel}
\usepackage{lmodern}
\usepackage[top=2cm,bottom=2cm,left=3cm,right=3cm]{geometry}
\usepackage{microtype}
\usepackage{mathtools, amssymb, amsthm}
\usepackage{mdframed}
\usepackage{hyperref}
\usepackage{graphicx}
\usepackage{xcolor}
\usepackage{mathrsfs}
\usepackage{wrapfig}
\usepackage{stmaryrd}


\newtheorem{prop}{Proposition}[section]
\newtheorem{theorem}{Théorème}
\newtheorem{definition}{Définition}[section]
\newtheorem*{remark}{Remarque}
\newtheorem*{lemma}{Lemme}
\newtheorem*{corollary}{Corollaire}
\newtheorem*{mth}{Méthode}
\newmdtheoremenv{thm}{Théorème}
\newtheorem{exo}{Exercice}
\newtheorem{exemple}{Exemple}


\newcommand*{\TakeFourierOrnament}[1]{{%
\fontencoding{U}\fontfamily{futs}selectfont\char#1}}
\newcommand*{\danger}{\TakeFourierOrnament{66}}

\newcommand{\lesss}{\rotatebox[origin=c]{90}{$\land$}}
\newcommand{\less}{\ \lesss\ }

\newcommand{\biggg}{\rotatebox[origin=c]{90}{$\lor$}}
\newcommand{\bg}{\ \biggg\ }

\title{\bsc{Analyse fonctionnelle et distributions}}
\date{2023-2024}

\begin{document}

\maketitle

\tableofcontents

\chapter{Espaces localement convexes}

\section{Rappels de topologie}

J. Dieudonné 1 et 2.

Reed-Simon 1, 2 et 4.

Brézis, ``Analyse fonctionnelle''

\

Soit $X$ ensemble. Soit $(X, \mathscr{T} )$ espace topologique où $\mathscr{T}  \subset \mathscr{P}(X) $.

$\mathscr{T} $ parcourt l'ensemble des voisinages de $x$ où $x$ est un point quelconque de $X$.

\subsection{Axiomes}

\begin{enumerate}
  \item Soient $x \in X$ et $ V'$ voisinage de $x$. Si $V \supset V'$ alors $V$ est un voisinage de $X$.
  \item $\bigcap_{\text{ finie } }  V_i$ est un voisinage de $x$, $\bigcap_{\text{ finie } } V_i \in \mathscr{T}  $,  mais $\bigcap_{\varepsilon \bg 0} V _{\varepsilon }  \neq \emptyset$ n'est pas un voisinage de $0$.
  \item $\bigcap_{i \in I} V_i $ est un voisinage de $x$.
\end{enumerate}


\begin{definition}[Ouvert]
  $\Omega$ ouvert si et seulement si $\Omega$ est voisinage de chacun de ses points.
\end{definition}


\begin{exemple}
  $(-1, 1)$ ouvert tandis que $[-1, 1)$ non ouvert car -1 n'a pas de voisinage.
\end{exemple}

\

$V(x) =(x- \varepsilon , x+ \varepsilon )$ est une base de voisinage pour la topologie usuelle de $\mathbb{R}$.

\begin{exo}
  On peut définir axiomatiquement $\mathscr{T} $ à partir de ses ouverts.
\end{exo}

\begin{definition}[Fermé]
  On dit que $F$ est un fermé si et seulement si $F ^{C}$ est un ouvert.
\end{definition}

\subsection{Cas particulier d'espaces topologiques : espaces métriques}



\begin{definition}[Espace métrique, distance]
  \

  $X$ est un ensemble, $d : X \times X \to \mathbb{R} ^{+}$ distance sur $X$ si et seulement si :
  \begin{enumerate}
    \item $d(x, y) = 0$ si et seulement si $x = y$ ;
    \begin{remark}
      Si on a seulement $x =y \implies d(x,y) =0$, alors $d$ est un écart.
    \end{remark}
    \item $d(x,y) = d(y,x)$ (symétrie) ;
    \item $d(x,y) \leq d(x,z) + d(z,y)$ (inégalité triangulaire).
    De ce fait, $\lvert d(x,z) - d(y,z) \rvert \leq d(x,y)$.
  \end{enumerate}
\end{definition}



\begin{exemple}
  \begin{enumerate}
    \item Dans $\mathbb{R}^n$, $d(x,y) = \Vert x-y \Vert $.
    \item $X$ ensemble. On définit $d$ de la façon suivante :
    $$ \forall x, y \in X,  d(x, y) = \begin{cases}
      d(x,y) =0 \text{ si } x=y \\
      d(x,y) = 1 \text{ si } x \neq y.
    \end{cases}$$

    Il s'agit de la distance triviale.

    Si $x,y,z$ distincts alors $d(x,y) \leq d(x,z)+d(z,y)$.
  \end{enumerate}

\end{exemple}


\subsection{Comparaison des topologies}

\

Soient $X$ un ensemble et $\mathscr{T}, \mathscr{T}' $ des topologies sur $X$.

\begin{definition}[Plus fine]
  On dit que $\mathscr{T}' $ est plus fine que $\mathscr{T} $ et on note $\mathscr{T}' \prec \mathscr{T}  $ si et seulement si $\mathscr{T} \subset \mathscr{T}' $.

  On dit aussi que $\mathscr{T}' $ est plus forte que $\mathscr{T} $.
\end{definition}

\begin{remark}
  Si $\mathscr{T}' \prec \mathscr{T}  $, il y a plus d'ouverts dans $\mathscr{T}' $ que dans $\mathscr{T} $ (idem pour les fermés).
\end{remark}

\begin{proof}
  Soit $\Omega$ ouvert dans $X$. On a $\Omega \in \mathscr{T}  \implies \Omega \in \mathscr{T}' $.

  Soit $F$ un fermé dans $X$. On a $F \in \mathscr{T}$, mais $ \Omega = F ^{C} \in \mathscr{T} \implies F ^{C} \in \mathscr{T}' $, donc $F \in \mathscr{T}' $.
\end{proof}

\paragraph{Formulations équivalentes}

\begin{enumerate}
  \item On suppose que $\mathscr{T}' \prec \mathscr{T} $. Si $\forall x \in X $, $U$ est un voisinage de  $x$ pour $\mathscr{T} $, alors $U $ voisinage de $x$ pour $\mathscr{T}' $, car si $U $ est un ouvert de $\mathscr{T} $, alors $U $ est un ouvert de $ \mathscr{T}' $.
  \item Pour l'application identité définie comme suit

  $$ id : (X, \mathscr{T}' ) \longrightarrow (X, \mathscr{T} ), $$

  %allant de $(X, \mathscr{T}')$ vers $(X, \mathscr{T} )$,

  on a $\mathscr{T}' \prec \mathscr{T}  $ si et seulement si $id $ est continue.
\end{enumerate}

\

Par exemple, prenons $X = \{ f :[0, 1] \to \mathbb{R} \} $. On prend $\mathscr{T} $ topologie de la convergence simple, i. e.

$f_n $ converge vers $ f$ simplement si $\forall x \in [0, 1], f_n(x) \to f(x)$.

\subparagraph{Ouverts de $\Omega$}

$$\Omega _{a, \varepsilon } = \{ f \in X \mid \sup _{i =1, \dots, k} \lvert f(a_i) \rvert \less \varepsilon \}, $$ avec $a = a_0, \dots, a_k \in [0, 1]$ et $\varepsilon  \bg 0$.

$\Omega _{a, \varepsilon } $ est un voisinage de 0 (la fonction nulle) dans $X$.

Pour $f_0 \in X$, $\Omega _{a, \varepsilon } + f_0$ est une base de voisinage de $f_0$, car $X$ est un espace vectoriel (on agit par translation).

\

On considère maintenant la topologie de la convergence uniforme $\mathscr{T}' $.

$\Omega _{\varepsilon } = \{ f \in X, \sup_{ x \in [0, 1] } \lvert f(x) \rvert \less \varepsilon   \} $ est un voisinage de $0$ (la fonction nulle).

\begin{prop}
  $\mathscr{T}' $ est plus fine que $\mathscr{T} $, ie $\mathscr{T} \subset \mathscr{T}' $.
\end{prop}

\begin{proof}
  Soit $\Omega _{a, \varepsilon } \in \mathscr{T}  $.

%  $f \in \Omega _{a, \varepsilon }$, alors on a $\lvert f(a_i) \rvert \less \varepsilon $.

  Si $f \in \Omega _{\varepsilon }$, alors $$\forall x \in [0, 1], \lvert f(x) \rvert \less \varepsilon, $$ ce qui implique que $$ \forall i \in \{ 1, \dots, k \}, \lvert f(a_i) \rvert \less \varepsilon \ (\text{car c'est vrai pour tout } x). $$

   Donc $\Omega  _{\varepsilon } $ est un voisinage de 0 dans $\mathscr{T} $. On a ainsi démontré que $\mathscr{T}' $ est plus fine que $\mathscr{T} $.

  %Or $\Omega _{a, \varepsilon } \subset \Omega _{\varepsilon }$, donc $\Omega _{ \varepsilon }$ est un voisinage de $0$ pour la topologie $\mathscr{T} $.
\end{proof}

\

On considère l'espace des fonctions continues $\mathcal{C}^0$ avec la norme $$ \Vert f \Vert _{0} = \sup_{  } \lvert f(x) \rvert $$ et l'espace des fonctions de classe $\mathcal{C}^1$ $\mathcal{C}^1$ avec la norme $$\Vert f \Vert _{1} = \sup \lvert f(x) \rvert + \sup \lvert f'(x) \rvert.$$

La topologie sur $\mathcal{C}^1$ est plus fine que celle sur $\mathcal{C}^0$.

\begin{proof}
  On a pour tout $f$, $$\Vert f \Vert _{0} \leq \Vert f \Vert _{1} .$$

  Ainsi si $$\Vert f \Vert _{1} \less \varepsilon,   $$ alors $$\Vert f \Vert _{0} \less \varepsilon.  $$

  Par conséquent, $\{ f , \Vert f \Vert _{1 } \less \varepsilon  \} \subset \{ f, \Vert f \Vert _{0} \less \varepsilon   \} $.

  Donc $\mathscr{T}' \prec \mathscr{T}  $.

  On sait également que si $U$ est un voisinage de 0 pour $\mathscr{T} $, alors $U$ est un voisinage de 0 pour $\mathscr{T}' $.
\end{proof}

\paragraph{Topologie métrisable (exemples)}

\begin{enumerate}
  \item Topologie grossière $\mathscr{T}= \{ \emptyset, X \} $. C'est la topologie la moins fine.

  \begin{remark}
    $\mathscr{T}' = \mathscr{P}(X)  $ est la topologie la plus fine.
  \end{remark}

  Vérifions si la topologie grossière est métrisable dans différents cas.
  \begin{itemize}
    \item Si $X = \{ a \} $, on a $d(a,a) =0$. Le seul voisinage de $a$ est $X = \{ a \} $. Donc $\mathscr{T} $ est métrisable.
    \item Supposons que $X = \{ a,b \} $. Mais $\mathscr{T} $ n'est plus métrisable, avec $d(a,b) =1$ (distance triviale).

    Raisonnons par l'absurde. Si $\mathscr{T} $ était métrisable, $\mathscr{T} $ devrait contenir un ouvert $\Omega$ tel que $a \in \Omega$ et $b \notin \Omega$. Or $\mathscr{T} = \{ \emptyset, X \} $, donc c'est impossible.
  \end{itemize}

  Pour $\mathscr{T}' $, on choisit la distance $d$ telle que $d(x,y) = 0 \text{ ou } 1$. Est-ce que $\mathscr{T}' $ est métrisable ?

  \item Prenons $\mathscr{T} $ telle que $\mathscr{T} = \{ \emptyset, \{ a \} , X \} $.

  On suppose que $X$ contient au moins deux éléments. Dans ce cas, $\mathscr{T} $ est une topologie sur $X$ non métrisable, car si $d(a,b)=1,$ avec $ b \neq a$, alors dans $\mathscr{T} $ il n'existe pas de boule ouverte qui contient $\{ b \} $ sans contenir $\{ a \} $.

  \item Considérons $X = \{ a,b \}$ muni de la topologie $  \mathscr{T} = \{ \emptyset, \{ a \}, \{ b \}, X \} = \mathscr{P}(X)  $.

  On a $d(a,b) = 1$, car $a \neq b$.

  De ce fait :

  $\{ a \} $ voisinage de $a$ qui ne contient pas $b$ ( $\{ a \} = \{ x \text{ tel que } d(x,a) \less 1 \} $) ;

  $\{ b \} $ voisinage de $b$ qui ne contient pas $a$.




\end{enumerate}

\subsection{Espaces vectoriels topologiques}

Dans le cas où $(X, \mathscr{T} )$ est un espace vectoriel topologique, il suffit de connaître les voisinages de 0 et on agit par translation pour déterminer les voisinages de n'importe quel $x \in X$.

\begin{definition}[Continuité]
  Soient $X, Y$ deux espaces vectoriels topologiques et $f: X \to Y$ une application. On considère :

  \begin{gather*}
    (U_a) _{a \in A} \text{ voisinage de } 0 \text{ dans } X \\
    (V_b) _{b \in B} \text{ voisinage de } 0 \text{ dans  } Y
  \end{gather*}

  $f$ est continue si pour tout $V = V_b + f(x_0)$ dans $Y$, il existe $U = \bigcap_{\text{finie} } (U_a +x_0) $ voisinage de $x$ dans $X$ tel que $x \in U \implies f(x) \in V$.
\end{definition}

\paragraph{Cas particulier : $X$ normé}

\begin{definition}[Norme]
  $\Vert \cdot \Vert $ est une norme sur $X$ si

  \begin{enumerate}
    \item $\Vert x \Vert = 0 \iff x=0 $ (séparation);
    \item $\Vert \lambda x \Vert = \lvert l \rvert \Vert x \Vert $ (absolue homogénéité);
    \item $\Vert x+y \Vert \leq \Vert x \Vert + \Vert y \Vert $ (inégalité triangulaire).
  \end{enumerate}
\end{definition}

De cette norme, on construit la distance $d$ telle que $$\forall x, y \in X, d(x,y) = \Vert x-y \Vert.$$

Voisinages de 0.

$(U_a) = B(0, a)$

$A = \mathbb{R} ^{+}$.

\begin{itemize}
  \item $f: X \to Y$ continue en $x_0$, $\forall V = V_b + f(x_0)$, $\exists U = B(0, \delta ) + x_0 , f(U) \subset V$.
  \item $X, Y$ EVN.

  $\forall \varepsilon  \bg 0, \exists \delta  \bg 0, f(B(0, \delta ) + x_0) \subset B(f(x_0), \varepsilon )$.
\end{itemize}

\marginpar{15-09-2023}

\section{Semi-normes et espaces localement convexes}

\subsection{Semi-normes sur $X$ espace vectoriel}

\begin{definition}[Semi-norme]
  L'application $\rho : X \to \mathbb{R} ^{+}$ est une semi-norme si :
  \begin{enumerate}
    \item $\rho(0) =0$ ;
    \item $\rho(\lambda x) = \lvert \lambda  \rvert \rho(x)$ ;
    \item $\rho(x+y) \leq \rho(x) + \rho(y)$.
  \end{enumerate}
\end{definition}

$X$ est un espace vectoriel $\mathbb{R}$ ou $\mathbb{C}$.

\begin{remark}
  {\fontencoding{U}\fontfamily{futs}\selectfont\char 66\relax} \ On n'a pas forcément $\rho(x) = 0 \implies x=0$.

\end{remark}


\begin{exemple}
  \begin{enumerate}
    \item Si $\rho$ est une norme, c'est aussi une semi-norme.
    \item $X = \mathcal{C}^0([0, 1], \mathbb{R} \ (\text{ou } \mathbb{C}))$. On prend $a = (a_0, \dots, a_k) \subset [0, 1]$. On définit

    \begin{equation*}
      \rho_a (f) = \sup_{ 0 \leq i \leq k } \lvert f(a_i) \rvert .
    \end{equation*}

    \item Topologie faible. $X$ est un espace vectoriel et $X'$ est son dual (espace contenant les formes linéaires sur $X$).

    Soit $l$ une forme linéaire dans $X'$. Alors

    \begin{equation*}
      p(x) = \lvert \langle l,x \rangle  \rvert.
    \end{equation*}
  \end{enumerate}
\end{exemple}



\begin{definition}[Famille de semi-normes séparée]
  Soit $(\rho_a) _{a \in A}$ une famille de semi-normes. On dit que $(\rho_a) _{a \in A}$ sépare les points (ou est séparée) si et seulement si

  \begin{equation*}
    \forall a \in A, \rho_a(x) = 0 \implies x=0.
  \end{equation*}
\end{definition}

\begin{definition}[Espace localement convexe (ELC)]
  $X$ est un espace localement convexe si et seulement si $X$ est muni d'une famille de semi-normes qui séparent les points.
\end{definition}

\begin{prop}
  Si $X$ est un espace localement convexe, alors $X$ est un espace vectoriel topologique pour la topologie définie par $\rho_a$.
\end{prop}

%\begin{remark}[Personnelle, n'est pas donnée dans le cours]
%  Un espace vectoriel topologique est dit localement convexe si
%\end{remark}

\begin{proof}
  On note $\mathscr{T} $ la topologie définie par la famille de semi-normes $(\rho_a) _{a \in A}$.

  \begin{remark}[Personnelle]
    On cherche à montrer que les $\mathcal{O} _{a, \varepsilon }$ forment une topologie. On va vérifier les axiomes de topologie.
  \end{remark}

  Dans ce cas, les ouverts $\mathcal{O} \in \mathscr{T} $ sont $\mathcal{O} = \mathcal{O} _{a, \varepsilon }, a \in A, \varepsilon  \bg 0$ définis ci-dessous :

  \begin{equation*}
    \mathcal{O} _{a, \varepsilon } = \{ x \mid \rho_a(x) \less \varepsilon  \}
  \end{equation*}

  $\mathcal{O} _{a,\varepsilon }$ une base de voisinages de 0.

  Les voisinages de $x$ sont donnés par translation :

  \begin{equation*}
    x + \mathcal{O} _{a, \varepsilon } = \{ x+y, y \in \mathcal{O} _{a, \varepsilon } \}.
  \end{equation*}

  On montre facilement que $\bigcap_{\text{finie} } \mathcal{O} _{a, \varepsilon } \in \mathscr{T} $ et $\bigcup_{\text{quelconque} } \mathcal{O} _{a, \varepsilon } \in \mathscr{T}  $.
\end{proof}

\begin{prop}
  $\mathscr{T} $ est la topologie la moins fine sur $X$ qui rend continues

  \begin{gather*}
    (x,y) \mapsto x+y \text{ et } (\lambda , x) \to \lambda x.
  \end{gather*}

  Il y a donc une compatibilité avec la structure des espaces vectoriels.
\end{prop}

\begin{proof}

  \begin{enumerate}
    \item   $\mathscr{T} $ rend continues les deux opérations de $X$. On a en effet

      \begin{gather*}
        \rho_a(x+y) \leq \rho_a(x)+ \rho_a(y).
      \end{gather*}

      Il suffit de prendre $\rho_a(x) \less \frac{\varepsilon }{2}$ et $\rho_a(x) \less \frac{\varepsilon }{2}$, on obtient $\rho_a(x+y) \less \varepsilon $.

      On a $\rho(\lambda x) = \lvert \lambda \rvert \rho(x)$ et on démontre ce résultat par analogie.

      \item La moins fine (en exercice).
  \end{enumerate}

\end{proof}

\begin{thm}\label{Hausdorff}
  La topologie de $X$ espace localement convexe est Hausdorff, i. e. elle sépare les points.
\end{thm}

\begin{definition}[Hausdorff]
  $(X, \mathscr{T} )$ est de Hausdorff si et seulement si pour tout $x, y \in X$ tel que $x \neq y$, il existe $\mathcal{O}_x$ et $\mathcal{O}_y$ voisinages de $x$ et de $y$ tels que $$ \mathcal{O}_x \cap \mathcal{O}_y = \emptyset.$$
  % tel que $y \notin \mathcal{O}_x$ et il existe $\mathcal{O}_y$ tel que $x \notin \mathcal{O}_y$.
\end{definition}


\begin{exemple}
  On prend $X = \{ a,b \}, \mathscr{T} = \{ \emptyset, \{ a \}, \{ b \}, X \}  $. On a $\{ a \} \cap \{ b \} = \emptyset $. Donc $(X, \mathscr{T} )$ est séparée.
\end{exemple}

\begin{proof}[Démonstration du théorème \ref{Hausdorff}]
  Par contraposée, on prend $y \neq 0$ et $ x = 0$.

  Si $X$ est un espace localement convexe, alors il existe $a \in A$ tel que $\rho_a(y) = \varepsilon \bg 0$.

  On pose
  \begin{gather}
    V_x = \left\{ z, \rho_a(z) \less \frac{\varepsilon }{2} \right\} \text{ et } V_y = \left\{ z, \rho(z-y) \less \frac{\varepsilon }{2} \right\} .
  \end{gather}

  Par l'inégalité triangulaire, on obtient $V_x \cap V_y = \emptyset$, car

  \begin{gather*}
    \rho_a(x-y) \bg \lvert \rho_a(x) - \rho_a(y) \rvert \bg \left\lvert \frac{\varepsilon }{2} - \varepsilon  \right\rvert = \frac{\varepsilon }{2} \bg 0.
  \end{gather*}

  %\begin{equation*}
  %  \rho_a(x-y) \bg
  %\end{equation*}

  %On a $y \in \mathcal{O} _{a, \varepsilon } + y$.

  %On pose $U_0 = \{ x, \rho_a(x) \less \frac{\varepsilon }{2}\} $.

  %Montrons que $x \notin \mathcal{O} _{a, \varepsilon } + y$.

  %On a

  %\begin{equation*}
    %\rho_a(x-y) \bg \varepsilon  - \frac{\varepsilon }{2} - \frac{\varepsilon }{2}.
  %\end{equation*}
\end{proof}

\section{Pourquoi ``localement convexe''?}

\begin{definition}
  Soit $X$ un $\mathbb{R}$ ou $\mathbb{C}$ espace vectoriel.

  \begin{enumerate}
    \item On dit que $C \subset X$ est convexe si

    \begin{equation*}
      \forall x, y \in C, \forall t \in  [0, 1], z = tx + (1-t)y \in C.
    \end{equation*}

    \item On dit que $B \subset X$ est balancé (sur $\mathbb{R}$) ou cerclé (sur $\mathbb{C}$) si

    \begin{equation*}
      \forall \lambda  \in \mathbb{R}, \lvert \lambda  \rvert = 1 \implies \forall x \in B, \lambda x \in B.
    \end{equation*}

    \item On dit que $E \subset X$ est équilibré si

    \begin{equation*}
      \forall \lambda  \in \mathbb{R} \text{ ou } \mathbb{C}, \lvert \lambda  \rvert \leq 1 \implies \forall x \in E, \lambda x \in E.
    \end{equation*}

    \item On dit que $A$ est absorbant si

    \begin{equation*}
      \bigcup _{t \bg 0} t A = X.
    \end{equation*}
  \end{enumerate}
\end{definition}

\begin{figure}[h!]
  \centering
  \includegraphics[scale=0.5]{figures/convexe1.png}
  \caption{Ensemble convexe}
  \label{}
\end{figure}



\begin{exemple}
  \begin{enumerate}
    \item Si $X$ est un espace vectoriel normé, $A = B(0, 1)$ et $x \in X$, on a $\frac{x}{\Vert x \Vert } \in B(0, 1)$. Alors $x \in \Vert x \Vert B(0, 1)$.
    \item Si $0 \in C$ convexe, alors $C$ est équilibré si et seulement si $C$ est balancé.
  \end{enumerate}
\end{exemple}


\begin{proof}
  On suppose que $C$ est balancé. Pour $x \in C \implies -x \in C$, donc $[-x, x] \in C$ par convexité.
\end{proof}

\begin{thm}
  Soit $X$ un espace vectoriel topologique. Les assertions suivantes sont équivalentes :

  \begin{enumerate}
    \item $X$ est un espace localement convexe (réel ou complexe);
    \item Il existe une base de voisinages de $0 \in X$ qui sont convexes, balancés (cerclés), absorbants.
  \end{enumerate}
\end{thm}

\marginpar{19-09-2023}

\begin{proof}
  \begin{enumerate}
    \item Si $X$ est un espace localement convexe, alors une base de voisinages de 0 est donnée par

    \[
    \mathcal{O} _{a, \varepsilon } = \{ x  \in X \mid \rho_a(x) \less \varepsilon  \}
    \]

    Les $\mathcal{O} _{a, \varepsilon }$ sont convexes, balancés et absorbants (TD).

    \item On utilise la jauge de Minkowski \ref{jauge_mink}.

    On pose \[
    \rho_C(x) = \mu_C(x).
    \]

    et on vérifie que $\rho_C$ est une semi-norme. Grâce au lemme \ref{lemme_mink}, on obtient les résultats suivants :

    \begin{enumerate}
      \item $\rho_C(x+y) \leq \rho_C(x)+ \rho_C(y)$, car $C$ est convexe ;
      \item $\rho_C(\lambda x) = \lambda \rho_C(x)$ si $\lambda \bg 0$ et $\rho_C(\lambda x) = \lvert \lambda  \rvert \rho_C(x)$, car $C$ est cerclé.

       %$\rho_C(\lambda x) = \rho_C((-\lambda )x)$ et on a

      %\[
      %\rho_C(\lambda x) = (-\lambda ) \rho_C(x) = \lvert \lambda  \rvert \rho_C(x),
      %\]

      %car $C$ est cerclé.

      $X$ muni de $\rho_C$ est un espace localement convexe.
    \end{enumerate}
  \end{enumerate}
\end{proof}

\begin{definition}[Jauge de Minkowski] \label{jauge_mink}
  Soit $X$ espace vectoriel réel ou complexe. On suppose que $C$ tel que $0 \in C$ est absorbant. Alors la jauge de Minkowski est définie comme suit :

  \[
  \mu_C(x) = \inf \{ \alpha \bg 0, x \in \alpha C \}.
  \]
\end{definition}

\begin{figure}[h!]
  \centering
  \includegraphics[scale=0.3]{figures/jaugem.png}
  \caption{La jauge de Minkowski}
  \label{}
\end{figure}

\begin{remark}
  Si $C$ est absorbant, alors $\forall x \in X$, $\mu_C(x) \less \infty$.
\end{remark}

\begin{lemma} \label{lemme_mink}
  Soit $C \subset X$ absorbant tel que $0 \in C$.
  \begin{enumerate}
    \item Si $\lambda  \geq 0$, $\mu_C(\lambda x)= \lambda \mu_C(x)$ ;
    \item Si $C$ est convexe, alors $\mu_C(x+y) \leq \mu_C(x)+ \mu_C(y)$ ;
    \item Si $C$ est cerclé, alors $\mu_C(\lambda x) = \lvert \lambda  \rvert \mu_C(x)$ ;
    \item $\{ x \in X, \mu_C(x) \less 1 \} \subset C \subset \{ x \in X, \mu_C(x) \leq 1 \} $.
  \end{enumerate}
\end{lemma}

\subsection{Théorème de Hahn-Banach}

Il y a la forme analytique et la forme géométrique de ce théorème.

\begin{thm}[De Hahn-Banach, forme analytique]
  Pour simplifier, on prend $X$ espace vectoriel sur $\mathbb{R}$. Soit $p : X \longrightarrow \mathbb{R}$ qui vérifie :
  \begin{itemize}
    \item [$\star$] $\forall x \in X, \forall \lambda \bg 0, p(\lambda x) = \lambda p(x)$ ;
    \item [$\star$] $\forall x, y \in X$, $p(x+y) \leq p(x) +p(y)$.
  \end{itemize}

  Soient $Y$ un sous espace vectoriel de $X$ et $l$ une forme linéaire sur $Y$ qui vérifie

  \[
  \forall x \in Y, l(x) \leq p(x), \forall x \in Y.
  \]

  Alors (prolongement) il existe $L$ forme linéaire sur $X$ telle que $L _{|Y} = l$ \textbf{et}

  \[
  \forall x \in X, L(x) \leq p(x).
  \]
\end{thm}

On l'applique aux espaces vectoriels normés, espaces localement convexes,...

\begin{thm}[Norme sur un espace dual]
  Soit $X$ espace vectoriel normé, $X'$ formes linéaires continues sur $X$, $X'$ est un espace vectoriel normé. La norme sur $X'$ est définie de la façon suivante :

  \[
  \Vert L \Vert _{X'} \stackrel{\text{déf}}{=} \sup_{ \substack{x \in X \\ \Vert x \Vert =1 } } \lvert \langle L,x \rangle  \rvert = \sup_{ x \in X \setminus \{ 0 \} } \left\lvert \left\langle L, \frac{x}{\Vert x \Vert } \right\rangle  \right\rvert = \sup_{ x \in X \setminus \{ 0 \} } \frac{\lvert \langle L,x \rangle  \rvert}{\Vert x \Vert }.
  \]

\end{thm}


\begin{exo}
  Montrer que $\Vert \cdot \Vert _{X'} $ est une norme.
\end{exo}

Si $X$ est un espace vectoriel normé complet (de Banach), alors $X'$ l'est aussi.

\begin{corollary}[Prolongement isométrique de $l$ sur $Y$]
  Soit $X$ espace vectoriel normé, $Y \subset X$ sous espace vectoriel de $X$ et $l \in Y'$ avec

  \[
  \Vert l \Vert = \sup_{ \substack{\Vert y \Vert \leq 1 \\ y \in Y} } \lvert \langle l,y \rangle  \rvert.
  \]

  Alors il existe un prolongement $L$ de $l$ de même norme

  \[
  \sup_{ \substack{x \in X \\ \Vert x \Vert \leq 1 } }  \lvert \langle l,x \rangle  \rvert = \sup_{ \substack{y \in Y \\ \Vert y \Vert \leq 1 } } \lvert \langle l,y \rangle  \rvert.
  \]
\end{corollary}

\begin{proof}
  Par le théorème de Hahn-Banach, on pose $p$ telle que $p(x) = \Vert l \Vert _{Y'} \Vert x \Vert  $ (l'application définie ainsi vérifie les propriétés de $p$ nécessaires à l'application du théorème).

  Par Hahn-Banach, il existe $L$ une forme linéaire sur $X$ telle que
  \[
  L(x)  = \langle L,x \rangle \leq p(x) = \Vert l \Vert _{Y'} \Vert x \Vert.
  \]

  Mais
  \[
  \langle L, -x \rangle \leq \Vert l \Vert _{Y'} \Vert -x \Vert,
  \]

  donc
  \[
  \lvert \langle L,x \rangle  \rvert \leq \Vert l \Vert _{Y'} \Vert x \Vert.
  \]

  Ainsi, en divisant par $\Vert x \Vert $, on obtient le résultat suivant :

  \[
  \forall x \in X, \left\lvert \left\langle L, \frac{x}{\Vert x \Vert } \right\rangle  \right\rvert \leq \Vert l \Vert _{Y'}.
  \]

  Or si on prend $x \in Y$,

  \[
  \left\lvert \left\langle L, \frac{x}{\Vert x \Vert } \right\rangle  \right\rvert \leq \Vert l \Vert _{Y'} = \sup_{ y \in Y \setminus \{ 0 \} } \left\lvert \left\langle L, \frac{y}{\Vert y \Vert } \right\rangle  \right\rvert.
  \]

  Comme $Y \subset X$ (ce qui entraîne que $\sup_{ y \in Y \setminus \{ 0 \} } \left\lvert \left\langle L, \frac{y}{\Vert y \Vert } \right\rangle   \right\rvert \leq \sup_{ x \in X \setminus \{ 0 \} } \left\lvert \left\langle L, \frac{x}{\Vert x \Vert } \right\rangle   \right\rvert  $), on a donc égalité, d'où l'isométrie.
\end{proof}

\begin{corollary}
  $\forall x_0 \in X$ espace vectoriel réel, il existe $L_0 \in X', \Vert L_0 \Vert _{X'} = \Vert x_0 \Vert _{X} $.
\end{corollary}

\begin{proof}
  $Y = \mathbb{R} x_0$. Soit $l(tx_0) \stackrel{\text{déf}}{=} t \Vert x_0 \Vert ^2 $ forme linéaire continue sur $Y$.

  Alors, en posant $t=1$, on obtient

  \[
  \Vert l \Vert _{Y'} =  \Vert x_0 \Vert
  \]

  %\left( \sup_{ y \in Y \setminus \{ 0 \} } \frac{\lvert \langle L,y \rangle  \rvert}{ \Vert y \Vert } = \frac{\Vert x_0 \Vert ^2 }{\Vert x_0 \Vert } \right) =

  et, par le théorème de Hahn-Banach,

  \[
  \Vert L_0 \Vert _{X'} = \Vert x_0 \Vert _{X}.
  \]
\end{proof}

\begin{exo}
  Traduire Hahn-Banach dans le cas où $X$ est un espace localement convexe.
\end{exo}

\begin{thm}[De Hahn-Banach, forme géométrique]
  Soit $X$ espace vectoriel normé (ou espace localement convexe). Soient $A, B \subset X$ convexes et disjoints.

  \begin{enumerate}
    \item On suppose que $A$ est ouvert. Alors il existe un hyperplan affine (d'équation $\langle L,x \rangle = \text{constante}$) $\mathscr{H} $ qui sépare au sens large $A$ et $B$.
    \item Si $A$ est fermé, $B$ est compact, alors il existe $\mathscr{H} $ hyperplan qui sépare $A$ et $B$ au sens strict.
  \end{enumerate}
\end{thm}

\begin{figure}[h!]
  \centering
  \includegraphics[scale=0.3]{figures/hb-geo.png}
  \caption{$A = \{ x_1 \less 0 \}, B = \{ x_2 \geq 0 \}, \mathscr{H} = \{ x_1 = 0 \}  $.}
  \label{}
\end{figure}




\end{document}
